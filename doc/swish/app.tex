% Copyright 2019 Beckman Coulter, Inc.
%
% Permission is hereby granted, free of charge, to any person
% obtaining a copy of this software and associated documentation files
% (the "Software"), to deal in the Software without restriction,
% including without limitation the rights to use, copy, modify, merge,
% publish, distribute, sublicense, and/or sell copies of the Software,
% and to permit persons to whom the Software is furnished to do so,
% subject to the following conditions:
%
% The above copyright notice and this permission notice shall be
% included in all copies or substantial portions of the Software.
%
% THE SOFTWARE IS PROVIDED "AS IS", WITHOUT WARRANTY OF ANY KIND,
% EXPRESS OR IMPLIED, INCLUDING BUT NOT LIMITED TO THE WARRANTIES OF
% MERCHANTABILITY, FITNESS FOR A PARTICULAR PURPOSE AND
% NONINFRINGEMENT. IN NO EVENT SHALL THE AUTHORS OR COPYRIGHT HOLDERS
% BE LIABLE FOR ANY CLAIM, DAMAGES OR OTHER LIABILITY, WHETHER IN AN
% ACTION OF CONTRACT, TORT OR OTHERWISE, ARISING FROM, OUT OF OR IN
% CONNECTION WITH THE SOFTWARE OR THE USE OR OTHER DEALINGS IN THE
% SOFTWARE.

\chapter {Developing Software with Swish}\label{chap:app}

\section {Introduction}

Swish can be used to build, test, and deploy programs ranging
from small scripts to large stand-alone applications.
This chapter describes some of the tools and mechanisms that Swish provides
for these purposes.

% \section {Theory of Operation}
% perhaps describe boot process?

\section {Deployment Types}

For interactive development, Swish provides a REPL that reads, evaluates,
and prints the values of programs entered at the prompt.
At the REPL, the \code{load} procedure can be used to evaluate the
contents of a file containing source or object code.
This is convenient when developing larger programs.

Swish provides several options for deploying programs.
This section describes these options and their trade-offs.

\subsection {Scripts}

A simple deployment option is to place source code in a file that begins with
a \code{\#!} line specifying the absolute path to an executable that can
evaluate the script.
This could be the absolute path to the Swish executable.
More often \code{/usr/bin/env} is used to locate the Swish executable via the
program search path.
For example, we might have:
\codebegin
$ cat hello
#!/usr/bin/env swish
(printf "Hello, World!{\textbackslash}n")
$ chmod +x hello
$ ./hello
Hello, World!
\codeend

In the preceding example, running \code{./hello} invokes the Swish executable
with \code{./hello} as its sole command-line argument.
At boot time, Swish calls \code{swish-start} to process its command-line arguments.
Since the first argument is not an option (\sopt{h}, \lopt{help}, etc.),
\code{swish-start} runs the file named by that argument with the
remaining arguments, if any, as its command-line arguments.
For example, the following \code{Echo} script processes the arguments
that are passed to the script.

\codebegin\fixtilde
$ cat Echo
#!/usr/bin/env swish
(printf "~\{~:(~a~)~\}{\textbackslash}n" (command-line-arguments))
$ chmod +x Echo
$ ./Echo some camel case identifiers are hard to read
SomeCamelCaseIdentifiersAreHardToRead
\codeend

To provide arguments to the Swish executable before the script filename, add
the \sopt{S} option to \code{env} and add the desired arguments after
\code{swish}.
Here the \code{-q} option suppresses the startup message and sets the prompt to
the empty string, and the \lopt{} option tells Swish to start a REPL after
loading the script.
\codebegin\fixtilde
$ cat howdy
#!/usr/bin/env -S swish -q \lopt{}
(printf "Howdy, Folks!{\textbackslash}n")
(printf "prompt: ~s{\textbackslash}n" (waiter-prompt-string))
(printf "command-line: ~s{\textbackslash}n" (command-line-arguments))
$ chmod +x howdy
$ ./howdy
Howdy, Folks!
prompt: ""
command-line: ("-q" "--" "./howdy")
(waiter-prompt-string "yes?")
yes? (+ 2 3)
5
yes? (exit)
\codeend

% to show boot search while loading script:
% #!/usr/bin/env -S swish --verbose

% Note that we *can* write a script named --help, since
% the argument comes in as something like ./--help.

\subsubsection {Limitations}\label{sec:script-limitations}

There are several constraints to consider when deploying Swish scripts. 
\begin{itemize}
  \item
    Naturally, Swish must be installed.
    To use \code{/usr/bin/env} as in the preceding examples, the \code{PATH}
    environment variable must contain the directory where the Swish executable
    is installed.
    This is preferable to hard-coding the absolute path to the Swish executable
    in your scripts.
  \item
    Chez Scheme must also be installed.
    In particular, the version of Chez Scheme that was used to build Swish
    must be installed.
    Swish must be able to locate the Chez Scheme boot files \code{petite.boot}
    and \code{scheme.boot}.
    % (Yes, we currently expect both boot files.)

    If Chez Scheme is installed in a non-standard location, it may be
    necessary to set the \code{SCHEMEHEAPDIRS} environment variable
    to help Swish locate the boot files.
    To see where Swish looks for boot files, run \code{swish \lopt{verbose}}.
  \item
    The \code{\#!} scripts shown in this section do not run under Windows.
    In MinGW/MSYS, these scripts may work unless Posix path conversion is
    disabled by setting the \code{MSYS\_NO\_PATHCONV} environment variable to 1.
    In Cygwin, these scripts may work if the directory containing the script
    is mounted as its Windows twin, e.g., \code{C:/Users} and \code{/Users}.
    For either of these options to work, Swish must be able to locate the
    appropriate Chez Scheme DLL via the standard search order.
    Scripts will fail if the current drive is not the drive containing the script.
\end{itemize}

\subsection {Linked Programs}

A linked program is simply a Scheme object file that begins
with a ``\code{\#!/usr/bin/env swish}'' line.
Swish runs these programs the same way it runs scripts,
except that \code{swish-start} skips the compilation step
when it runs the file.

We can use \code{swish-build} to build a linked program \code{foo} from a
source file \code{foo.ss}, as follows:
\codebegin
$ swish-build -o foo foo.ss
\codeend

\subsubsection {Limitations}

All of the limitations in Section~\ref{sec:script-limitations} apply.
Since linked programs use \code{/usr/bin/env}, the Swish executable
must be in a directory in the \code{PATH} on the target machine.

In addition, the Swish executable (and supporting code) that is used
to run a linked program must be identical to the one that was used to
build it.

\subsection {Stand-alone Programs}

A stand-alone program consists of an executable and a boot file
that must be installed in the same directory.
The name of the boot file is the same as the executable's name
with any extension replaced by \code{.boot}.
The executable is simply a copy of the Swish executable.
The boot file includes the necessary Chez Scheme boot files
along with the compiled application code.

We can use \code{swish-build} to build a linked program \code{foo} from a
source file \code{foo.ss} by specifying a base boot file via the \sopt{b}
option:
\codebegin
$ swish-build -o foo foo.ss -b petite
\codeend

\subsubsection {Limitations}

On Windows, the Swish DLLs \code{osi.dll}, \code{uv.dll}, and \code{sqlite3.dll}
must be installed in the same directory as the program executable.
These DLLs can be found in the directory containing the Swish executable.
In addition, the program executable must be able to locate
Microsoft's C Runtime Library \code{vcruntime140.dll}
and the appropriate Chez Scheme DLL via the standard search order.

\subsection {Services}

On Linux and Windows, a Swish application can be started as a service
that listens for system shutdown, suspend, and resume messages.
To start a Swish application as a service, pass \code{/SERVICE} as
the first command-line argument.
On Windows, two additional command-line arguments are required:
the service name and the path to the log file where stdout and
stderr are redirected.
See \hyperref[swishservice]{\code{swish\_service}} for details.

% Rationale:
% - /SERVICE is not likely to be a Swish script on Unix system
% - /SERVICE is not likely to be a regular command-line argument
%   since (swish cli) uses - and -- for options
% - having a single executable
%   - means fewer binaries to manage
%   - makes it easy to start as a service or start interactively for debugging

\section{Running Tests}

Use the \code{mat} and \code{isolate-mat} forms described in
Section~\ref{sec:mat} to define automated tests.
Use the \code{swish-test} script to run tests and report the results.
This script treats each file with a \code{.ms} extension as a suite of tests.
See \code{swish-test \lopt{help} all} for details.

% TODO eventually describe
%  - #!eof mats

\section {Programming Interface}

\subsection{Configuration}

\defineentry{app:config}
\begin{procedure}
  \code{(app:config \opt{\var{obj}})}
\end{procedure}
\returns{} a hashtable

When called with no arguments, the \code{app:config} procedure
returns the configuration data cached in a private process parameter.
If the cache is empty, \code{app:config} first populates the cache by reading
data from the file identified by \code{app:config-filename}.
If that file does not exist, the procedure returns an empty hashtable.
Otherwise, \code{app:config} expects the file to contain a single JSON object.

The optional \var{obj} must be \code{\#f} or a hashtable in the form returned by
\code{json:make-object}.
Calling \code{app:config} with \code{\#f} clears the cache.
Calling \code{app:config} with a hashtable installs the hashtable as the cached value.

\defineentry{app:config-filename}
\begin{procedure}
  \code{(app:config-filename)}
\end{procedure}
\returns{} the name of the application's configuration file

The \code{app:config-filename} procedure returns the name of the application
configuration file that \code{app:config} will read.
The filename returned depends on the value of \code{app:path}.

If \code{app:path} is set to a value of the form
\code{"\var{...}/bin/\var{app-name}\opt{.\var{ext}}"},
that ends with a \code{"bin"} directory,
then the result is a \code{config} file in the corresponding \code{"lib"} directory
\code{"\var{...}/lib/\var{app-name}/config"}.
If \code{app:path} is set to a value of the form
\code{"\var{...}/\var{app-name}\opt{.\var{ext}}"},
then the result is a file with a \code{.config} extension in the same directory:
\code{"\var{...}/\var{app-name}.config"}.

If \code{app:path} is not set, the result is a \code{.config} file in the base
directory identified by value of the \code{base-dir} parameter.
This can be useful when loading program code at the REPL during interactive
development.

\defineentry{app:name}
\begin{parameter}
  \code{(app:name \opt{\var{name}})}
\end{parameter}
\hasvalue{} a string or \code{\#f}

The \code{app:name} parameter returns the short descriptive name of the
application, if known.
This value is used by the \code{app-exception-handler} to identify the program
that is reporting an error.
The value is also useful as an argument to \code{display-usage} and
\code{display-help}.

When called with a string, \code{app:name} treats the value as
a file-system path and stores only the last element of the path,
dropping any file extension.

For stand-alone programs, \code{swish-start}
sets this parameter to the path of the running executable.
For scripts or linked programs, \code{swish-start}
sets this parameter to the path of the script or program file.

\defineentry{app:path}
\begin{parameter}
  \code{(app:path \opt{\var{path}})}
\end{parameter}
\hasvalue{} an absolute path or \code{\#f}

The \code{app:path} parameter returns the absolute path to the script or
program, if known.
This value is used by \code{app:config-filename} to determine the
location of the application's configuration file.
For scripts and linked programs, where the actual executable is Swish, this
value is different from that returned by \code{osi\_get\_executable\_path}.

When called with a string, \code{app:path} calls \code{get-real-path} to
obtain an absolute path and stores the result.

\defineentry{base-dir}
\begin{parameter}
  \code{(base-dir \opt{\var{path}})}
\end{parameter}
\hasvalue{} a file-system path

When called without arguments, \code{base-dir} returns the file-system
path of the application's base directory.  Otherwise, the base
directory is set to \var{path}, which must specify the file-system
path of a directory that exists and is writable by the application.
Setting \code{base-dir} sets the values of \code{data-dir},
\code{log-file}, and \code{tmp-dir} so that they refer to locations
within the base directory.

When the application starts, \code{base-dir} is set to the current directory.

\defineentry{data-dir}
\begin{parameter}
  \code{(data-dir \opt{\var{path}})}
\end{parameter}
\hasvalue{} a file-system path

The \code{data-dir} parameter holds the file-system path of a directory
in which the application can write persistent data.
Applications should not assume that this directory exists, but should
instead use \code{make-directory-path} when creating files under \code{data-dir}.
Its initial value is the path to a \code{"data"} subdirectory of
the directory specified by \code{base-dir}.

\defineentry{log-file}
\begin{parameter}
  \code{(log-file \opt{\var{filename}})}
\end{parameter}
\hasvalue{} a file name

The \code{log-file} parameter holds the name of the file containing the SQLite
log database managed by the \code{log-db} gen-server, which creates the file
and the requisite directory structure when started.
Its initial value is the path to a \code{"Log.db3"} file in the directory
specified by \code{data-dir}.

\defineentry{tmp-dir}
\begin{parameter}
  \code{(tmp-dir \opt{\var{path}})}
\end{parameter}
\hasvalue{} a file-system path

The \code{tmp-dir} parameter holds the file-system path of a directory
in which the application can write ephemeral data.
Applications should not assume that this directory exists, but should
instead use \code{make-directory-path} when creating files under \code{tmp-dir}.
Its initial value is the path to a \code{"tmp"} subdirectory of
the directory specified by \code{base-dir}.

\defineentry{include-line}
\begin{syntax}
  \code{(include-line \var{filenme} \opt{\var{not-found}})}
\end{syntax}
\returns{} see below

The \code{include-line} macro expects a string constant \var{filename}
identifying a file either via absolute path or as a path relative to a
directory in \code{source-directories}.
If such a file can be found at expand time, \code{include-line} expands to a
string containing the first line of that file or the end-of-file object if the
file is empty.
If the file cannot be found, \code{include-line} expands to
\code{(\var{not-found} \var{filename})}, where \var{not-found} defaults to an
expression that raises a syntax error indicating that the file could not be
found.

\defineentry{software-info}
\begin{procedure}
  \code{(software-info)}
\end{procedure}
\hasvalue{} a JSON object

The \code{software-info} procedure returns a JSON object containing the values
stored by
\code{software-product-name},
\code{software-revision}, and
\code{software-version}.
Swish populates these parameters for the
\code{swish} and \code{chezscheme} keys
with values that are determined at compile time.

\defineentry{software-product-name}
\begin{parameter}
  \code{(software-product-name \opt{\var{key}} \opt{\var{value}})}
\end{parameter}
\hasvalue{} a string or \code{\#f}

The \code{software-product-name} procedure stores or retrieves the product
name under the path \code{(\var{key} product-name)} in the
\code{(software-info)} object.
If \var{value} is omitted, this procedure returns the value, if any,
associated with \var{key}, or else \code{\#f}.
If \var{key} and \var{value} are omitted, this procedure returns the value
associated with the key \code{(string->symbol (or (app:name) "swish"))}.
If specified, \var{key} must be a symbol and \var{value} must be a string.
If \var{key} and \var{value} are specified and a value has already been
set for \var{key}, this procedure has no effect.

\defineentry{software-revision}
\begin{parameter}
  \code{(software-revision \opt{\var{key}} \opt{\var{value}})}
\end{parameter}
\hasvalue{} a string or \code{\#f}

The \code{software-revision} procedure stores or retrieves the software
revision under the path \code{(\var{key} revision)} in the
\code{(software-info)} object.
If \var{value} is omitted, this procedure returns the value, if any,
associated with \var{key}, or else \code{\#f}.
If \var{key} and \var{value} are omitted, this procedure returns the value
associated with the key \code{(string->symbol (or (app:name) "swish"))}.
If specified, \var{key} must be a symbol and \var{value} must be a string.
If \var{key} and \var{value} are specified and a value has already been
set for \var{key}, this procedure has no effect.

\defineentry{software-version}
\begin{parameter}
  \code{(software-version \opt{\var{key}} \opt{\var{value}})}
\end{parameter}
\hasvalue{} a string or \code{\#f}

The \code{software-version} procedure stores or retrieves the software
version under the path \code{(\var{key} version)} in the
\code{(software-info)} object.
If \var{value} is omitted, this procedure returns the value, if any,
associated with \var{key}, or else \code{\#f}.
If \var{key} and \var{value} are omitted, this procedure returns the value
associated with the key \code{(string->symbol (or (app:name) "swish"))}.
If specified, \var{key} must be a symbol and \var{value} must be a string.
If \var{key} and \var{value} are specified and a value has already been
set for \var{key}, this procedure has no effect.

\subsection{Program Life Cycle}

\defineentry{app-exception-handler}
\begin{procedure}
  \code{(app-exception-handler \var{obj})}
\end{procedure}
\returns{} unspecified

When a Swish application starts, it sets the default value of the
\code{base-exception-handler} parameter to \code{app-exception-handler}.
This procedure expects a single argument, typically a condition or an object
passed to \code{throw} or \code{raise}, which it saves in the parameter \code{debug-condition}.

If \code{app:name} is set, then \code{app-exception-handler} writes a
message to the console error port, prefixed with the application name,
before calling \code{reset}.
If \var{obj} is a condition, then the message is formed by stripping the
``Warning:'' or ``Exception:'' prefix from the output of \code{display-condition}.
Otherwise, the message is generated by calling
\code{exit-reason->english}
on \var{obj}.

If \code{app:name} is not set, then \code{app-exception-handler}
calls the default exception handler.

\defineentry{app-sup-spec}
\begin{parameter}
  \code{(app-sup-spec \opt{\var{start-specs}})}
\end{parameter}
\hasvalue{} a list of child specifications (see page~\pageref{page:child-spec})

The \code{app-sup-spec} parameter supplies the \var{start-specs}
that define the tree of child processes that are started by
\code{app:start} and supervised by the \code{main-sup} gen-server.
The initial value of \code{app-sup-spec} is constructed by calling
\code{make-swish-sup-spec} with \code{swish-event-logger} as the
sole logger.

\defineentry{app:shutdown}
\begin{syntax}
  \code{(app:shutdown \opt{\var{exit-code}})}
\end{syntax}

This is an alias for \code{application:shutdown}.

\defineentry{app:start}
\begin{procedure}
  \code{(app:start)}
\end{procedure}
\returns{} \code{ok}

The \code{app:start} procedure calls \code{application:start}
with a \var{starter} thunk that calls \code{supervisor:start\&link}
with the value of the \code{app-sup-spec} parameter.

\defineentry{make-swish-sup-spec}
\begin{procedure}
  \code{(make-swish-sup-spec \var{loggers})}
\end{procedure}
\returns{} a list of child specifications (see page~\pageref{page:child-spec})

The \code{make-swish-sup-spec} procedure builds a list of child specifications
representing the default supervision tree described in
Section~\ref{sec:default-supervision-tree}.
The \var{loggers} argument is a list of \code{<event-logger>} tuples that will
be passed to \code{log-db:setup} when the \code{log-db} gen-server is started.

\defineentry{swish-start}
\begin{procedure}
  \code{(swish-start \var{arg} \etc)}
\end{procedure}
\returns{} see below

When a stand-alone program starts, it initializes several parameters:
\code{app:name}, \code{app:path}, \code{command-line}, and
\code{command-line-arguments}.
When a Swish script, linked program, or REPL is started, it performs this
initialization by applying \code{swish-start} to the original command-line
arguments.

The \code{swish-start} procedure expects zero or more strings as arguments
and scans them left to right.
It adds each string that begins with a single dash (\sopt{}) or double dash
(\lopt{}) to a set of options to be handled later.
If it encounters the string \str{\lopt{}}, it sets the REPL option and
saves the remaining arguments as files to be loaded.
If it encounters a string that does not begin with a dash, it stops scanning
the arguments and marks that string and those that follow as ordinary
arguments.

If \code{swish-start} finds \lopt{help} among the options,
it displays a summary of Swish's command-line options and returns.
If it finds \lopt{version} among the options,
it displays version information and returns.
If the REPL option is set or it finds no ordinary arguments, then it
suppresses the startup message and waiter prompt 
if \lopt{quiet} is among the options,
then loads any files that were specified,
and then calls \code{new-cafe}. 
Otherwise, it treats the first ordinary argument as a script or linked program.
In this case, it installs the ordinary arguments as
the value of \code{command-line},
sets \code{app:name} and \code{app:path} to the first of them,
sets \code{command-line-arguments} to the remaining ones,
and clears \code{app:config}
before loading the script or linked program.
If an error occurs while loading the script or linked program,
Swish exits with an exit code of 1.
Otherwise, Swish exits normally.

When it starts a REPL, script, or linked program, \code{swish-start} attempts
to import any libraries found in \code{library-list} whose library path begins
with the symbol \code{swish}.

\defineentry{repl-level}
\begin{parameter}
  \code{repl-level}
\end{parameter}
\hasvalue{} nonnegative fixnum

The \code{repl-level} parameter returns the nesting depth of the
\code{swish-start} REPL.
The value of this parameter is initially zero; it is incremented when a
process enters the \code{swish-start} REPL.
It is not affected by \code{new-cafe}.

\subsection{Foreign Interface}

Scheme libraries that rely on shared objects must arrange
to call \code{load-shared-object} before calling \code{foreign-procedure} or
calling \code{make-ftype-pointer} on a function ftype.
We must consider shared-object naming conventions and search paths on
different platforms when calling \code{load-shared-object}.
We can provide its filename argument through some combination of conditional
compilation, hard-coded relative or absolute paths, and general computation.

The \code{provide-shared-object} and \code{require-shared-object} procedures
in this section offer a simple way to:
factor platform dependencies out of client code,
load shared objects by absolute path to avoid search,
specify shared object file names as paths relative
to the application configuration file, and
hook the operation that loads shared object code.

These procedures coordinate via the application configuration object stored in
\code{app:config}, which is populated on demand by reading the file specified
by \code{app:config-filename}.
A Scheme library can call \code{require-shared-object} before its
\code{foreign-procedure} definitions and rely on demand-loading of the
configuration file to supply the absolute path to the shared object code when
the library is invoked.
This simplifies a process that can otherwise be complicated
by the fact that Chez Scheme invokes a library lazily as soon as
one of its exports may be referenced.

To see how these procedures are decoupled, consider the following Swish REPL
transcript on a 64-bit Linux machine:
\codebegin
> (provide-shared-object 'libc "/lib64/libc.so.6")
> (foreign-entry? "fork")
#f
> (require-shared-object 'libc)
> (foreign-entry? "fork")
#t
> (let ([op (open-file-to-write (app:config-filename))])
    (on-exit (close-port op)
      (json:write op (app:config) 0)))
> (exit)
\codeend

If we start a new Swish REPL, we can call \code{require-shared-object}
without first calling \code{provide-shared-object}
because we explicitly wrote the value of \code{app:config} to the location
specified by \code{app:config-filename} before exiting the original REPL.
\codebegin
> (foreign-entry? "fork")
#f
> (require-shared-object "libc")
> (foreign-entry? "fork")
#t
> (display (utf8->string (read-file (app:config-filename))))
\{
  "swish": \{
    "shared-object": \{
      "libc": \{
        "a6le": \{
          "file": "/lib64/libc.so.6"
        \}
      \}
    \}
  \}
\}
\codeend

\defineentry{provide-shared-object}
\begin{procedure}
  \code{(provide-shared-object \var{so-name} \var{filename})}
\end{procedure}
\returns{} unspecified

The \code{provide-shared-object} procedure expects a symbol
\var{so-name} as the generic name of a shared object and
a string \var{filename} that is the
name of the actual file containing the shared object code.
This procedure records \var{filename} in \code{app:config} under a
path that includes \code{so-name} and the host \code{machine-type}.

\defineentry{require-shared-object}
\begin{procedure}
  \code{(require-shared-object \var{so-name} \opt{\var{handler}})}
\end{procedure}
\returns{} unspecified

The \code{require-shared-object} procedure loads a shared object via an
absolute path to prevent \code{load-shared-object} from searching for the
shared object file in a system-specific set of directories.

The \var{so-name} argument is a symbol that specifies the generic name of a
shared object.
The optional \var{handler} is a procedure that expects three arguments:
\var{filename} is the absolute path to the shared object file,
\var{key} is \var{so-name}, and \var{dict} is the hashtable
retrieved from \code{app:config} via a path that includes
\var{so-name} and the host \code{machine-type}.
This hashtable contains a key \code{file} whose value
is the file name originally supplied by \code{provide-shared-object}.
The handler might examine other keys within the hashtable before
loading the shared object.
The default handler simply calls \code{load-shared-object} on \var{filename}.

If the file name retrieved from \code{app:config} is a relative path,
then \code{require-shared-object} determines the absolute path by
treating the file name as a path relative to the parent directory
containing the application configuration file name returned by \code{app:config-filename}.

\defineentry{define-foreign}
\begin{syntax}
  \code{(define-foreign \var{name} (\var{arg-name} \var{arg-type}) \etc)}
\end{syntax}
\expandsto{} \antipar\codebegin
(begin
  (define \var{name}* (foreign-procedure (symbol->string '\var{name}) (\var{arg-type} \etc) ptr))
  (define (\var{name} \var{arg-name} \etc)
    (match (\var{name}* \var{arg-name} \etc)
      [(,who . ,err)
       (guard (symbol? who))
       (io-error '\var{name} who err)]
      [,x x])))
\codeend

The \code{define-foreign} macro defines two procedures: \code{\var{name}*}
is a raw foreign procedure that expects the specified argument types and returns
a \code{ptr} value, while \var{name} is a wrapper that calls \code{\var{name}*}
and raises an \code{io-error} if it returns an error pair.

% Copyright 2018 Beckman Coulter, Inc.
%
% Permission is hereby granted, free of charge, to any person
% obtaining a copy of this software and associated documentation files
% (the "Software"), to deal in the Software without restriction,
% including without limitation the rights to use, copy, modify, merge,
% publish, distribute, sublicense, and/or sell copies of the Software,
% and to permit persons to whom the Software is furnished to do so,
% subject to the following conditions:
%
% The above copyright notice and this permission notice shall be
% included in all copies or substantial portions of the Software.
%
% THE SOFTWARE IS PROVIDED "AS IS", WITHOUT WARRANTY OF ANY KIND,
% EXPRESS OR IMPLIED, INCLUDING BUT NOT LIMITED TO THE WARRANTIES OF
% MERCHANTABILITY, FITNESS FOR A PARTICULAR PURPOSE AND
% NONINFRINGEMENT. IN NO EVENT SHALL THE AUTHORS OR COPYRIGHT HOLDERS
% BE LIABLE FOR ANY CLAIM, DAMAGES OR OTHER LIABILITY, WHETHER IN AN
% ACTION OF CONTRACT, TORT OR OTHERWISE, ARISING FROM, OUT OF OR IN
% CONNECTION WITH THE SOFTWARE OR THE USE OR OTHER DEALINGS IN THE
% SOFTWARE.

\subsection {Testing}

The \code{(swish mat)} library provides methods to
define, iterate through, and run test cases, and to log the
results.
The \code{swish-test} script provides a convenient way to
run tests and report the results.
See \code{swish-test \lopt{help} all} for details.
To access the \code{(swish mat)} library directly, run
\code{swish-test \lopt{repl}} instead of \code{swish},
then import the library as usual.

Test cases are called \emph{mats} and consist of a name, a
set of tags, and a test procedure of no arguments.
The set of mats is stored in
reverse order in a single, global list.
The list of tags allows the
user to group tests or mark them.  For example, tags can be used to
note that a test was created for a particular change request.

% ----------------------------------------------------------------------------
\defineentry{mat}\label{sec:mat}
\begin{syntax}
  \code{(mat \var{name} (\var{tag} \etc) $e_1$ $e_2$ \etc)}
\end{syntax}
\expandsto{} \code{(add-mat '\var{name} '(\var{tag} \etc)
  (lambda () $e_1$ $e_2$ \etc))}

The \code{mat} macro creates a mat with the given \var{name},
\var{tag}s, and test procedure $e_1$ $e_2$ \etc\ using the
\code{add-mat} procedure.

% ----------------------------------------------------------------------------
\defineentry{isolate-mat}
\begin{syntax}
\code{(isolate-mat \var{name} (\var{tag} \etc) $e_1$ $e_2$ \etc)}\\
\code{(isolate-mat \var{name} (settings [\var{key} \var{val}] \etc) $e_1$ $e_2$ \etc)}\strut
\end{syntax}
\expandsto{}
\code{(mat \var{name} (\var{tag} \etc) (\$isolate-mat (lambda () $e_1$ $e_2$ \etc)))}

Tests involving process operations, such as \code{spawn}, \code{send}, and
\code{receive}, should use \code{isolate-mat} in place of \code{mat}
to isolate the host system from the test code.
The \code{isolate-mat} macro is provided by the \code{(swish testing)} library,
which can be accessed via \code{swish-test}.

The \code{settings} form can be used within \code{isolate-mat} to override the
following defaults:

\begin{tabular}{lcp{.5\textwidth}}
  key & default & description \\ \hline
  \code{tags}    & \code{()} & a list of \var{tag} \etc \\
  \code{timeout} & 60000 & deadline for test to complete, in milliseconds;
                           computed as \code{(scale-timeout 'isolate-mat)} \\
  \code{process-cleanup-deadline} & 500 & maximum time to wait for spawned processes to terminate, in milliseconds \\
  \code{process-kill-delay} & 100 & delay in milliseconds before killing each spawned process that did not terminate
  by the \code{process-cleanup-deadline}
\end{tabular}

% ----------------------------------------------------------------------------
\defineentry{default-timeout}
\begin{procedure}
\code{(default-timeout \var{timeout} \opt{\var{value}})}
\end{procedure}
\returns{} see below

Given a symbol \var{timeout} and a \var{value}, which
must be a nonnegative fixnum or the symbol \code{infinity},
\code{default-timeout} associates \var{timeout} with \var{value}
and returns an unspecified value.
If the optional \var{value} is omitted, \code{default-timeout} returns the
value currently associated with \var{timeout}, or raises an exception if none.
If \var{timeout} is a fixnum, \code{default-timeout} is the identity function.
The following table lists predefined timeouts and their associated values.

\begin{tabular}{lll}
  timeout & value & use \\ \hline
  \code{infinity} & \code{infinity} & \\
  \code{isolate-mat} & 60000 & default \code{timeout} for \code{isolate-mat} \\
  \code{os-process}  & 10000 & used internally by some Swish tests that call \code{spawn-os-process}
\end{tabular}

% ----------------------------------------------------------------------------
\defineentry{scale-timeout}\index{timeout@\texttt{TIMEOUT\_SCALE\_FACTOR}}
\begin{procedure}
  \code{(scale-timeout \var{timeout})}
\end{procedure}
\returns{} see below

The \code{scale-timeout} procedure resolves \var{timeout} by calling
\code{(default-timeout \var{timeout})}.
If the result is the symbol \code{infinity}, it is returned unmodified.
If the result is a fixnum, \code{scale-timeout} returns the product
of that number and the scale factor obtained by converting
the value of the \verb|TIMEOUT_SCALE_FACTOR| environment variable to a number
via \code{string->number}.
The scale factor defaults to 1 and must be nonnegative.
The product is rounded to an exact number.

% ----------------------------------------------------------------------------
\defineentry{mat:add-annotation"!}
\begin{syntax}
  \code{(mat:add-annotation! \var{expr})}
\end{syntax}
\returns{} unspecified

The \code{mat:add-annotation!} macro extends the list of annotations recorded in
the \code{meta-data} for the mat result of the current mat.
Each annotation records the \code{erlang:now} timestamp,
the source for the call site, if available, and
the value of \var{expr}, which must evaluate to a valid JSON datum.
This macro must be called only within the dynamic extent of a running mat.
See \hyperref[load-results]{\code{load-results}} for more information.

% ----------------------------------------------------------------------------
\defineentry{add-mat}
\begin{procedure}
  \code{(add-mat \var{name} \var{tags} \var{test})}
\end{procedure}
\returns{} unspecified

The \code{add-mat} procedure adds a mat to the front of the global
list. \var{name} is a symbol, \var{tags} is a list, and \var{test} is
a procedure of no arguments.

If \var{name} is already used, an exception is raised.

% ----------------------------------------------------------------------------
\defineentry{run-mat}
\begin{procedure}
  \code{(run-mat \var{name} \var{reporter})}
\end{procedure}
\returns{} see below

The \code{run-mat} procedure runs the mat of the given \var{name} by
executing its test procedure with an altered exception handler. If the
test procedure completes without raising an exception, the mat result
is \code{pass}. If the test procedure raises exception \var{e}, the
mat result is \code{(fail~.~\var{e})}.

After the mat completes, the \code{run-mat} procedure tail calls
\code{(\var{reporter} \var{name} \var{tags} \var{result} \var{statistics})}.

If no mat with the given \var{name} exists, an exception is raised.

% ----------------------------------------------------------------------------
\defineentry{run-mats}
\begin{syntax}
  \code{(run-mats \opt{\var{name}} \etc)}
\end{syntax}
\returns{} unspecified

The \code{run-mats} macro runs each mat specified by symbols
\var{name} \etc.  When no names are supplied, all
mats are executed.  After each mat is executed, its result, name, and
exception if it failed are displayed.  When the mats are finished, a
summary of the number run, passed, and failed is displayed.

% ----------------------------------------------------------------------------
\defineentry{run-mats-to-file}
\begin{procedure}
  \code{(run-mats-to-file \var{filename})}
\end{procedure}
\returns{} unspecified

The \code{run-mats-to-file} procedure executes all mats and writes
the results into the file specified by the string \var{filename}. If
the file exists, its contents are overwritten. The file format is a
sequence of JSON objects readable with \code{load-results} and
\code{summarize}.

% ----------------------------------------------------------------------------
\defineentry{for-each-mat}
\begin{procedure}
  \code{(for-each-mat \var{procedure})}
\end{procedure}
\returns{} unspecified

The \code{for-each-mat} procedure calls \code{(\var{procedure}
  \var{name} \var{tags})} for each mat, in no particular order.

% ----------------------------------------------------------------------------
\defineentry{load-results}
\begin{procedure}
  \code{(load-results \var{filename})}
\end{procedure}
\returns{} a JSON object
\phantomsection\label{load-results}

The \code{load-results} procedure reads the contents of the file
specified by string \var{filename} and returns a JSON object
with the following keys:

\begin{tabular}{lp{3.6in}}
  \code{meta-data} & a JSON object \\
  \code{report-file} & \var{filename} \\
  \code{results} & a list of JSON objects \\
\end{tabular}

The \code{meta-data} object contains at least the following keys:

\begin{tabular}{lp{4.6in}}
  \code{completed} & \code{\#t} if test suite completed, \code{\#f} otherwise \\
  \code{hostname} & \code{(osi\_get\_hostname)} of the host system \\
  \code{machine-type} & \code{(machine-type)} of the host system \\
  \code{test-file} & name of the file containing the tests \\
  \code{test-run} & uuid generated by \code{swish-test} for the set of tests run \\
  \code{uname} & the result from \code{get-uname} as a JSON object containing
    \code{os-machine},
    \code{os-release},
    \code{os-system}, and
    \code{os-version} \\
\end{tabular}

If the test suite completed, the \code{meta-data} object also contains the
following keys:

\begin{tabular}{lp{4.6in}}
  \code{date} & \code{(format-rfc2822 (current-date))} at the start of the test suite \\
  \code{software-info} & \code{(software-info)} for the tested code \\
  \code{timestamp} & \code{(erlang:now)} at the start of the test suite \\
\end{tabular}

Each result is a JSON object with the following keys:

\newcolumntype{L}[1]{>{\raggedright\let\newline\\\arraybackslash\hspace{0pt}}p{#1}}
\begin{tabular}{lp{4.6in}}
  \code{message} & error message from failing test, or empty string \\
  \code{meta-data} & a JSON object that contains: \\
  & \begin{tabular}{lL{3.5in}}
      \code{annotations} & an ordered list of JSON objects recording data from
        \code{mat:add-annotation!}, \\
      \code{start-time} & \code{(erlang:now)} at the start of the test, and \\
      \code{end-time} & \code{(erlang:now)} at the end of the test \\
    \end{tabular} \\
  \code{sstats} & a JSON object representing the \code{sstats-difference} for the test \\
  \code{stacks} & if test failed with exception \var{e}, then\\
                & \code{(map stack->json (exit-reason->stacks \var{e}))} \\
  \code{tags} & a list of strings corresponding to the symbolic tags in the \code{mat} form \\
  \code{test} & a string corresponding to the symbolic \code{mat} name \\
  \code{test-file} & the name of the test file \\
  \code{type} & the type of result: \code{"pass"}, \code{"fail"}, \code{"skip"} \\
\end{tabular}

% ----------------------------------------------------------------------------
\defineentry{summarize}
\begin{procedure}
  \code{(summarize \var{files})}
\end{procedure}
\returns{} five values: the number of passing mats, the number of
failing mats, the number of skipped mats, the number of completed suites,
and the length of \var{files}.

The \code{summarize} procedure reads the contents of each file in
\var{files}, a list of string filenames, and returns the number of
passing mats, the number of failing mats, the number of skipped mats,
the number of completed test suites, and the number of files specified.
An error is raised if any
entry is malformed.

