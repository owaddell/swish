% Copyright 2018 Beckman Coulter, Inc.
%
% Permission is hereby granted, free of charge, to any person
% obtaining a copy of this software and associated documentation files
% (the "Software"), to deal in the Software without restriction,
% including without limitation the rights to use, copy, modify, merge,
% publish, distribute, sublicense, and/or sell copies of the Software,
% and to permit persons to whom the Software is furnished to do so,
% subject to the following conditions:
%
% The above copyright notice and this permission notice shall be
% included in all copies or substantial portions of the Software.
%
% THE SOFTWARE IS PROVIDED "AS IS", WITHOUT WARRANTY OF ANY KIND,
% EXPRESS OR IMPLIED, INCLUDING BUT NOT LIMITED TO THE WARRANTIES OF
% MERCHANTABILITY, FITNESS FOR A PARTICULAR PURPOSE AND
% NONINFRINGEMENT. IN NO EVENT SHALL THE AUTHORS OR COPYRIGHT HOLDERS
% BE LIABLE FOR ANY CLAIM, DAMAGES OR OTHER LIABILITY, WHETHER IN AN
% ACTION OF CONTRACT, TORT OR OTHERWISE, ARISING FROM, OUT OF OR IN
% CONNECTION WITH THE SOFTWARE OR THE USE OR OTHER DEALINGS IN THE
% SOFTWARE.

\chapter {Generic Server}\label{chap:gen-server}

\begin{quote}
  \emph{The generic server provides an ``empty'' server, that is, a
  framework from which instances of servers can be
  built.} \hfill ---Joe~Armstrong~\cite{armstrong-thesis}
\end{quote}

\section {Introduction}

In a concurrent system, many processes need to access a shared
resource or sequentially manipulate the state of the system. This is
generally modeled using a client/server design pattern. To help
developers build robust servers, a generic server
(\code{gen-server}) implementation inspired by Erlang's Open Telecom
Platform is provided.

The principles of the generic server can be found in Joe Armstrong's
thesis~\cite{armstrong-thesis} or \emph{Programming Erlang---Software
  for a Concurrent World}~\cite{programming-erlang}. Documentation for
Erlang's \code{gen\_server} is available
online~\cite{gen-server-ref}. Source code for the Erlang Open Telecom
Platform can be found online~\cite{erlang}. The source code for
\code{gen\_server} is part of stdlib and can be found in
/lib/stdlib/src/gen\_server.erl.

\section {Theory of Operation}

A \code{gen-server} provides a consistent mechanism for programmers
to create a process which manages state, timeout conditions, and
failure conditions using functional programming techniques. A
programmer uses \code{gen-server:start\&link} and implements the callback
API to instantiate particular behavior.

A generic server starts a new process, registers it as a named
process, and invokes the \code{init} callback procedure while
blocking the calling process.

Clients can then send messages to a server using the synchronous
\code{gen-server:call}, the asynchronous \code{gen-server:cast},
or the raw \code{send} procedure. The \code{gen-server} framework
will automatically process messages and dispatch them to
\code{handle-call}, \code{handle-cast}, and \code{handle-info}
respectively.

The \code{gen-server} framework code automatically interprets a
\code{stop} return value from the callback API or an \code{EXIT}
message from the process which created it as a termination request and
calls \code{terminate}. If the termination reason satisfies the
\code{informative-exit-reason?} predicate, generic servers use
\code{event-mgr:notify} to report the termination.

Erlang's \code{gen\_server} supports timeouts during
\code{gen\_server:start} and \code{gen\_server:start\&link}. In
order to simplify the startup code, we have not implemented this
feature. Timeouts while running the \code{init} callback may cause
resources to be stranded until the garbage collector can clean them
up. Timeouts during initialization should be considered carefully.

\section {Programming Interface}

\defineentry{gen-server:start\&link}
\begin{syntax}
  \code{(gen-server:start\&link \var{name} \var{arg} \etc)}
\end{syntax}
\returns{}
\code{\#(ok \var{pid})\alt{}%
  \#(error \var{reason})\alt{}%
  ignore}

\begin{argtbl}
  \argrow{name}{a symbol for a registered server or \code{\#f} for
    an anonymous server}
  \argrow{arg}{any Scheme datum}
\end{argtbl}

\code{gen-server:start\&link} spawns the server process, links to
the calling process, registers the server process as \var{name}, and calls
\code{(init \var{arg} ...)} within that process. To ensure a
synchronized startup procedure, \code{gen-server:start\&link} does
not return until \code{init} has returned.

This macro uses the current scope to capture the callback functions
\code{init}, \code{handle-call}, \code{handle-cast},
\code{handle-info}, and \code{terminate}.

Attempting to register a name that already exists results in
\code{\#(error \#(name-already-registered \var{pid}))}, where
\var{pid} is the existing process.

The return value of \code{gen-server:start\&link} is propagated from
the \code{init} callback.

An \code{init} which returns \code{\#(ok \var{state}
  \opt{\var{timeout}})} will yield \code{\#(ok \var{pid})} where
\var{pid} is the newly created process.

An \code{init} which returns \code{\#(stop \var{reason})} or exits
with \var{reason} will terminate the process and yield
\code{\#(error \var{reason})}.

An \code{init} which returns \code{ignore} will terminate the
process and yield \code{ignore}. This value is useful to inform a
supervisor that the \code{init} procedure has determined that this
server is not necessary for the system to operate.

An \code{init} which returns \var{other} values will terminate the
process and yield \code{\#(error \#(bad-return-value \var{other}))}.

\defineentry{gen-server:start}
\begin{syntax}
  \code{(gen-server:start \var{name} \var{arg} \etc)}
\end{syntax}
\returns{}
\code{\#(ok \var{pid})\alt{}%
  \#(error \var{error})\alt{}%
  ignore}

\code{gen-server:start} behaves the same as
\code{gen-server:start\&link} except that it does not link to the
calling process.

\defineentry{gen-server:enter-loop}
\begin{syntax}
  \code{(gen-server:enter-loop \var{state} \opt{\var{timeout}})}
\end{syntax}
\returns{} does not return

\code{gen-server:enter-loop} transforms the calling process into a
generic server. The \var{state} and \var{timeout} are equivalent to
those returned by \code{init}.

On entry, the macro calls \code{(process-name)} to determine the
registered name of the process, if any, and \code{(process-parent)} to
determine the spawning process, if available, for logging and
processing termination. As a result, the name recorded in
\code{<gen-server-terminating>} will not reflect subsequent changes in
process registration.

This macro uses the current scope to capture the callback functions
\code{handle-call}, \code{handle-cast}, \code{handle-info}, and
\code{terminate}.

While \code{gen-server:enter-loop} does not return normally, it does
raise an exception upon termination. This allows any exception
handlers or winders on the stack to run.

\defineentry{gen-server:call}
\begin{procedure}
  \code{(gen-server:call \var{server} \var{request} \opt{\var{timeout}})}
\end{procedure}
\returns{}
\var{reply}

\begin{argtbl}
  \argrow{server}{process or registered name}
  \argrow{request}{any Scheme datum}
  \argrow{timeout}{non-negative exact integer in milliseconds or
    \code{infinity}, defaults to 5000}
\end{argtbl}

\code{gen-server:call} sends a synchronous \var{request} to
\var{server} and waits for a \var{reply}. The server processes the
request using \code{handle-call}.

Failure to receive a reply causes the calling process to exit with
reason \code{\#(timeout \#(gen-server call (\var{server}
  \var{request})))} if no timeout is specified, or \code{\#(timeout
  \#(gen-server call (\var{server} \var{request} \var{timeout})))} if
a timeout is specified. If the caller catches the failure and
continues running, the caller must be prepared for a possible late
reply from the server.

When the reply is a fault condition, the fault is thrown in the
calling process.

\code{gen-server:call} exits if the server terminates while the
client is waiting for a reply. When that happens, the client exits for
the same reason as the server.

\defineentry{gen-server:cast}
\begin{procedure}
  \code{(gen-server:cast \var{server} \var{request})}
\end{procedure}
\returns{}
\code{ok}

\begin{argtbl}
  \argrow{server}{process or registered name}
  \argrow{request}{any Scheme datum}
\end{argtbl}

\code{gen-server:cast} sends an asynchronous \var{request} to a
\var{server} and returns \code{ok} immediately. When using
\code{gen-server:cast} a client does not expect failures in the
server to cause failures in the client; therefore, this procedure
ignores all failures. The server will process the request using
\code{handle-cast}.

\defineentry{gen-server:reply}
\begin{procedure}
  \code{(gen-server:reply \var{client} \var{reply})}
\end{procedure}
\returns{}
\code{ok}

\begin{argtbl}
  \argrow{client}{a \var{from} argument provided to the
    \code{handle-call} callback}
  \argrow{reply}{any Scheme datum}
\end{argtbl}

A server can use \code{gen-server:reply} to send a
\var{reply} to a \var{client} that called \code{gen-server:call}
and is blocked awaiting a reply.

In some situations, a server cannot reply immediately to a client.
In such cases, \code{handle-call} may store the \var{from} argument
and return \code{no-reply}. Later, the server can call
\code{gen-server:reply} using that \var{from} value as
\var{client}. The \var{reply} is the return value of the
\code{gen-server:call} in this case.

\defineentry{gen-server:debug}
\begin{procedure}
  \code{(gen-server:debug \var{server} \var{server-options} \var{client-options})}
\end{procedure}
\returns{} \code{ok}

\begin{argtbl}
  \argrow{server}{process or registered name}
  \argrow{server-options}{\code{(\opt{message} \opt{state}
      \opt{reply})\alt{}\#f}}
  \argrow{client-options}{\code{(\opt{message} \opt{reply})\alt{}\#f}}
\end{argtbl}

\code{gen-server:debug} sets the debugging mode of \var{server}.
The \var{server-options} argument specifies the logging of calls in
the server. When \var{server-options} is \code{\#f}, server logging
is turned off. Otherwise, server logging is turned on, and
\var{server-options} is a list of symbols specifying the level of
detail.  In logging mode, the \var{server} sends a
\code{<gen-server-debug>} event for each call to
\code{handle-call}, \code{handle-cast}, and
\code{handle-info}. The \var{message} field is populated when
\code{message} is in \var{server-options}, the \var{state} field is
populated when \code{state} is in \var{server-options}, and the
\var{reply} field is populated when \code{reply} is in
\var{server-options}.

Similarly, the \var{client-options} argument specifies the logging of
client calls to \var{server} with \code{gen-server:call}.  When
\var{client-options} is \code{\#f}, client logging is turned
off. Otherwise, client logging is turned on, and \var{client-options}
is a list of symbols specifying the level of detail. In logging mode,
\code{gen-server:call} sends a \code{<gen-server-debug>}
event. The \var{message} field is populated when \code{message} is
in \var{client-options}, and the \var{reply} field is populated when
\code{reply} is in \var{client-options}.

\defineentry{define-state-tuple}
\begin{syntax}
  \code{(define-state-tuple \var{name} \var{field} \etc)}
\end{syntax}

This form defines a tuple type using \code{(define-tuple
  \var{name} \var{field} \etc)} and defines a new syntactic form
\code{\$state}. \code{\$state} provides a succinct syntax for the
\code{state} variable.

\code{\$state} transforms \code{(\$state \var{op} \var{arg} \etc)}
to \code{(\var{name} \var{op} state \var{arg} \etc)} where
\code{state} is a variable in the same scope as \code{\$state}.

Given this definition:

\code{(define-state-tuple <my-state> x y z)}

The following code is equivalent:

\begin{tabular}{l l}
  \code{(<my-state> copy state [x 2])} & \code{(\$state copy [x 2])} \\
  \code{(<my-state> x state)} & \code{(\$state x)} \\
  \code{(<my-state> y state)} & \code{(\$state y)} \\
  \code{(<my-state> z state)} & \code{(\$state z)} \\
\end{tabular}

There is no equivalent for constructing a state tuple because
constructing a tuple does not require the \code{state}
variable. The \code{(<my-state> make \etc)} syntax must be used.

\section {Published Events}

All generic servers send the event manager the following event:

\begin{pubevent}{<gen-server-terminating>}
  \argrow{timestamp}{the time the event occured}
  \argrow{name}{the name of the server}
  \argrow{pid}{the server process}
  \argrow{last-message}{the last message received by the server}
  \argrow{state}{the last state passed into \code{terminate}}
  \argrow{reason}{the reason for termination}
  \argrow{details}{\code{\#f} or a fault-condition containing the reason for termination}
\end{pubevent}

This event is fired after a successful call to \code{terminate}
if the reason for termination satisfies the \code{informative-exit-reason?}
predicate.
If the \code{terminate} procedure exits with a
new \var{reason}, the event contains the new \var{reason}.

\begin{pubevent}{<gen-server-debug>}
  \argrow{timestamp}{the time the operation started}
  \argrow{duration}{the duration of the operation in milliseconds}
  \argrow{type}{1 for \code{handle-call}, 2 for \code{handle-cast}, 3
    for \code{handle-info}, 4 for \code{terminate}, 5 for
    a successful \code{gen-server:call}, and 6 for a failed
    \code{gen-server:call}}
  \argrow{client}{the client process or \code{\#f}}
  \argrow{server}{the server process}
  \argrow{message}{the message sent to the server or \code{\#f}}
  \argrow{state}{the state of the server when it received the message
    or \code{\#f}}
  \argrow{reply}{the server's reply or \code{\#f}}
\end{pubevent}

\section {Callback Interface}

A programmer implements the callback interface to define a particular
server's behavior. All callback functions are called from within the
server process.

The callback functions for gen-server processes are supposed to be
\emph{well-behaved functions}, i.e., functions that work
correctly. The generation of an exception in a well-behaved function
is interpreted as a failure~\cite{armstrong-thesis}.

When a callback function exits with a reason, \code{terminate} is
called and the server exits.

When a callback function returns an unexpected \var{value},
\code{terminate} is called with the reason
\code{\#(bad-return-value \var{value})}, and the server exits.

A callback may specify a \var{timeout} as a relative time in
milliseconds up to one day, an absolute time in milliseconds (e.g.,
from \code{erlang:now}), or \code{infinity}. The default
\var{timeout} is \code{infinity}. If the time period expires before
another message is received, then a \code{timeout} message will be
processed by \code{handle-info}.

Messages sent using \code{send}, including those matching
\code{`(EXIT \var{pid} \var{reason})}
and \code{`(DOWN \var{monitor} \var{pid} \var{reason})},
are processed by \code{handle-info}.

The generic server framework will automatically interpret an
\code{EXIT} message from the process which spawned it as a reason
for termination. \code{terminate} will be called
directly. \code{handle-info} will not be called. The server must use
\code{(process-trap-exit \#t)} to receive \code{EXIT} messages.

\defineentry{init}
\begin{procedure}
  \code{(init \var{arg} \etc)}
\end{procedure}
\returns{}
\code{\#(ok \var{state} \opt{\var{timeout}})\alt{}%
  \#(stop \var{reason})\alt{}%
  ignore}

\begin{argtbl}
  \argrow{arg \etc}{the \var{arg} \etc{} provided to
    \code{gen-server:start\&link} or \code{gen-server:start}}
  \argrow{state}{any Scheme datum}
  \argrow{timeout}{relative time in milliseconds up to one day,
    absolute time in milliseconds (e.g., from \code{erlang:now}), or
    \code{infinity} (default)}
  \argrow{reason}{any Scheme datum}
\end{argtbl}

\code{(init \var{arg} \etc)} is called from a new server process
started by a call to \code{gen-server:start\&link} or \code{gen-server:start}.
Calls to those procedures block until \code{init} returns.

A successful \code{init} returns \code{\#(ok \var{state}
  \opt{\var{timeout}})}. The \var{state} is then maintained
functionally by the generic server framework.

\code{init} may specify that server initialization failed by
returning \code{\#(stop \var{reason})}. The server will then fail to
start using this \var{reason}. \code{terminate} will not be called
as the server has not properly started.

\code{init} may return \code{ignore}. The server will then exit
with reason \code{normal}, and \code{gen-server:start\&link} will
return \code{ignore}. This is used to inform a
supervisor that the server is not necessary
for the system to operate. \code{terminate} will not be called.

\defineentry{handle-call}
\begin{procedure}
  \code{(handle-call \var{request} \var{from} \var{state})}
\end{procedure}
\returns{}
\code{\#(reply \var{reply} \var{state} \opt{\var{timeout}})\alt{}%
  \#(no-reply \var{state} \opt{\var{timeout}})\alt{}%
  \#(stop \var{reason} \opt{\var{reply}} \var{state})}

\begin{argtbl}
  \argrow{request}{the \var{request} provided to \code{gen-server:call}}
  \argrow{from}{\code{\#(\var{client-process} \var{tag})}}
  \argrow{state}{server state}
  \argrow{reply}{any Scheme datum}
  \argrow{timeout}{relative time in milliseconds up to one day,
    absolute time in milliseconds (e.g., from \code{erlang:now}), or
    \code{infinity} (default)}
  \argrow{reason}{any Scheme datum}
\end{argtbl}

\code{handle-call} is responsible for processing a client
\var{request} generated by \code{gen-server:call}.

\code{handle-call} may return \code{\#(reply \var{reply}
  \var{state} \opt{\var{timeout}})} to indicate that \var{reply} is to
be returned from \code{gen-server:call} to the caller. The server
state will become \var{state}.

\code{handle-call} may return \code{\#(no-reply \var{state}
  \opt{\var{timeout}})} to continue operation and to indicate that the
caller of \code{gen-server:call} will continue to wait for a
reply. The server state will become \var{state}. The server will need
to use \code{gen-server:reply} and \var{from} to reply to the
client.

\code{handle-call} may return \code{\#(stop \var{reason}
  \opt{\var{reply}} \var{state})} to set a new \var{state}, then
terminate the server with the given \var{reason}. If the optional
\var{reply} is specified, it will be the return value of
\code{gen-server:call}; otherwise, \code{gen-server:call} will
exit with \var{reason}.

\var{reply} is any Scheme datum.

\var{state} is any Scheme datum.

\var{reason} is any Scheme datum.

\defineentry{handle-cast}
\begin{procedure}
  \code{(handle-cast \var{request} \var{state})}
\end{procedure}
\returns{}
\code{\#(no-reply \var{state} \opt{\var{timeout}})\alt{}%
  \#(stop \var{reason} \var{state})}

\begin{argtbl}
  \argrow{request}{the \var{request} provided to \code{gen-server:cast}}
  \argrow{state}{server state}
  \argrow{timeout}{relative time in milliseconds up to one day,
    absolute time in milliseconds (e.g., from \code{erlang:now}), or
    \code{infinity} (default)}
  \argrow{reason}{any Scheme datum}
\end{argtbl}

\code{handle-cast} is responsible for processing a client
\var{request} generated by \code{gen-server:cast}.

\code{handle-cast} may return \code{\#(no-reply \var{state}
  \opt{\var{timeout}})} to continue operation. The server state will
become \var{state}.

\code{handle-cast} may return \code{\#(stop \var{reason}
  \var{state})} to terminate the server with the given
\var{reason}. The server state will become \var{state}.

\defineentry{handle-info}
\begin{procedure}
  \code{(handle-info \var{msg} \var{state})}
\end{procedure}
\returns{}
\code{\#(no-reply \var{state} \opt{\var{timeout}})} $|$
\code{\#(stop \var{reason} \var{state})}

\begin{argtbl}
  \argrow{msg}{\code{timeout} or a Scheme datum sent via \code{send}}
  \argrow{state}{server state}
  \argrow{timeout}{relative time in milliseconds up to one day,
    absolute time in milliseconds (e.g., from \code{erlang:now}), or
    \code{infinity} (default)}
  \argrow{reason}{any Scheme datum}
\end{argtbl}

\code{handle-info} is responsible for processing timeouts and
miscellaneous messages sent to the server via \code{send}.

\code{handle-info} may return \code{\#(no-reply \var{state}
  \opt{\var{timeout}})} to continue operation. The server state will
become \var{state}.

\code{handle-info} may return \code{\#(stop \var{reason}
  \var{state})} to terminate the server with the given
\var{reason}. The server state will become \var{state}.

\defineentry{terminate}
\begin{procedure}
  \code{(terminate \var{reason} \var{state})}
\end{procedure}
\returns{} ignored

\begin{argtbl}
  \argrow{reason}{shutdown reason}
  \argrow{state}{server state}
\end{argtbl}

\code{terminate} is called when the server is about to terminate. It
is responsible for cleaning up any resources that the server
allocated. When it returns, the server exits for the given
\var{reason}.

\var{reason} can be any reason specified by a stop return value
\code{\#(stop \etc)}. When a supervision tree is terminating,
\var{reason} will be \code{shutdown}.

The return value of \code{terminate} is ignored.  If the
termination reason satisifies the \code{informative-exit-reason?}
predicate, the
generic server framework uses \code{event-mgr:notify} to report the
termination. The server then terminates for that reason.

If \code{terminate} exits with \var{reason}, then that reason is
logged, and the server terminates with \var{reason}.
