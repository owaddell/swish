% Copyright (c) 1999-2015, Dorai Sitaram.
% All rights reserved.

% Permission to copy, modify, distribute, and use this work or
% a modified copy of this work, for any purpose, is hereby
% granted, provided that the copy includes this copyright
% notice, and in the case of a modified copy, also includes a
% notice of modification.  This work is provided as is, with
% no warranty of any kind.

% Modifications are copyright 2020 Beckman Coulter, Inc.

% Permission is hereby granted, free of charge, to any person
% obtaining a copy of this software and associated documentation files
% (the "Software"), to deal in the Software without restriction,
% including without limitation the rights to use, copy, modify, merge,
% publish, distribute, sublicense, and/or sell copies of the Software,
% and to permit persons to whom the Software is furnished to do so,
% subject to the following conditions:
%
% The above copyright notice and this permission notice shall be
% included in all copies or substantial portions of the Software.
%
% THE SOFTWARE IS PROVIDED "AS IS", WITHOUT WARRANTY OF ANY KIND,
% EXPRESS OR IMPLIED, INCLUDING BUT NOT LIMITED TO THE WARRANTIES OF
% MERCHANTABILITY, FITNESS FOR A PARTICULAR PURPOSE AND
% NONINFRINGEMENT. IN NO EVENT SHALL THE AUTHORS OR COPYRIGHT HOLDERS
% BE LIABLE FOR ANY CLAIM, DAMAGES OR OTHER LIABILITY, WHETHER IN AN
% ACTION OF CONTRACT, TORT OR OTHERWISE, ARISING FROM, OUT OF OR IN
% CONNECTION WITH THE SOFTWARE OR THE USE OR OTHER DEALINGS IN THE
% SOFTWARE.

\chapter {Regular Expressions}\label{chap:pregexp}

\section {Introduction}

The regular expressions library \code{(swish pregexp)} is a derivative
of pregexp: Portable Regular Expressions for Scheme and Common
Lisp~\cite{pregexp}. It provides regular expressions modeled on
Perl's~\cite{friedl:regex,pperl} and includes such powerful directives
as numeric and non-greedy quantifiers, capturing and non-capturing
clustering, POSIX character classes, selective case- and
space-insensitivity, back-references, alternation, backtrack pruning,
positive and negative look-ahead and look-behind, in addition to the
more basic directives familiar to all regexp users.

A \emph{regexp} is a string that describes a pattern.  A regexp
matcher tries to \emph{match} this pattern against (a portion of)
another string, which we will call the \emph{text string}.  The text
string is treated as raw text and not as a pattern.

Most of the characters in a regexp pattern are meant to match
occurrences of themselves in the text string.  Thus, the pattern
\code{"abc"} matches a string that contains the characters \code{a},
\code{b}, \code{c} in succession.

In the regexp pattern, some characters act as \emph{metacharacters},
and some character sequences act as \emph{metasequences}.  That is,
they specify something other than their literal selves.  For example,
in the pattern \code{"a.c"}, the characters \code{a} and \code{c} do
stand for themselves but the \emph{metacharacter} `\code{.}'  can
match \emph{any} character (other than newline).  Therefore, the
pattern \code{"a.c"} matches an \code{a}, followed by \emph{any}
character, followed by a \code{c}.

If we needed to match the character `\code{.}' itself, we
\emph{escape} it, i.e., precede it with a backslash
(\code{\textbackslash}).  The character sequence
\code{{\textbackslash}.} is thus a \emph{metasequence}, since it
doesn't match itself but rather just `\code{.}'.  So, to match
\code{a} followed by a literal `\code{.}' followed by \code{c}, we use
the regexp pattern
\code{"a{\textbackslash}{\textbackslash}.c"}.\footnote{The double
  backslash is an artifact of Scheme strings, not the regexp pattern
  itself.  When we want a literal backslash inside a Scheme string, we
  must escape it so that it shows up in the string at all. Scheme
  strings use backslash as the escape character, so we end up with two
  backslashes.} Another example of a metasequence is
\code{{\textbackslash}t}, which is a readable way to represent the tab
character.

We will call the string representation of a regexp the
\emph{U-regexp}, where \emph{U} can be taken to mean {\em Unix-style}
or \emph{universal}, because this notation for regexps is universally
familiar.  Our implementation uses an intermediate tree-like
representation called the \emph{S-regexp}, where \emph{S} can stand
for \emph{Scheme}, \emph{symbolic}, or \emph{s-expression}.  S-regexps
are more verbose and less readable than U-regexps, but they are much
easier for Scheme's recursive procedures to navigate.

\section {Programming Interface}

% ----------------------------------------------------------------------------
\defineentry{pregexp}
\begin{procedure}
  \code{(pregexp \var{regexp})}
\end{procedure}
\returns{} an S-regexp

The \code{pregexp} procedure takes a U-regexp string \var{regexp} and
returns an S-regexp.

% ----------------------------------------------------------------------------
\defineentry{re}
\begin{syntax}
  \code{(re \var{regexp})}
\end{syntax}
\expandsto{}
\code{(pregexp \var{regexp})}

If \var{regexp} is a literal string, the \code{re} macro expands to the
result of evaluating \code{(pregexp\ \var{regexp})} at expand time.
Otherwise it expands into a run-time call to \code{pregexp}.

% ----------------------------------------------------------------------------
\defineentry{pregexp-match-positions}
\begin{procedure}
  \code{(pregexp-match-positions \var{pat} \var{str} \opt{\var{start} \opt{\var{end}}})}
\end{procedure}
\returns{} \code{((\var{s}~.~\var{e}) \etc)} or \code{\#f}

The \code{pregexp-match-positions} procedure takes a regexp pattern
\var{pat} and a text string \var{str} and returns a \emph{match} if
the regexp matches (some part of) the text string between the
inclusive \var{start} index (defaults to 0) and the exclusive
\var{end} index (defaults to the length of \var{str}).

The regexp may be either a U- or an S-regexp.
\code{pregexp-match-positions} will internally compile a U-regexp to
an S-regexp before proceeding with the matching.  If you find yourself
calling \code{pregexp-match-positions} repeatedly with the same
U-regexp, it may be advisable to explicitly convert the latter into an
S-regexp once beforehand, using \code{pregexp}, to save needless
recompilation.

\code{pregexp-match-positions} returns a list of \emph{index pairs} if
the regexp matches the string and \code{\#f} if it does not
match. Index pair \code{(\var{s}~.~\var{e})} gives the inclusive
starting index \var{s} and exclusive ending index \var{e} of the
matching substring with respect to \var{str}. The first index pair
indicates the entire match, and subsequent pairs indicate
submatches. Some of the submatches may be \code{\#f}.

% ----------------------------------------------------------------------------
\defineentry{pregexp-match}
\begin{procedure}
  \code{(pregexp-match \var{pat} \var{str} \opt{\var{start} \opt{\var{end}}})}
\end{procedure}
\returns{} list of matching substrings or \code{\#f}

The \code{pregexp-match} procedure is called like
\code{pregexp-match-positions}, but instead of returning index pairs,
it returns the matching substrings. The first substring is the entire
match, and subsequent substrings are submatches, some of which may be
\code{\#f}.

% ----------------------------------------------------------------------------
\defineentry{pregexp-split}
\begin{procedure}
  \code{(pregexp-split \var{pat} \var{str})}
\end{procedure}
\returns{} list of substrings from \var{str}

The \code{pregexp-split} procedure takes two arguments, a regexp
pattern \var{pat} and a text string \var{str}, and returns a list of
substrings of the text string, where the pattern identifies the
delimiter separating the substrings. The returned substrings do not
include the delimiter.

If the pattern can match an empty string, then the list of all the
single-character substrings is returned.

To identify one or more spaces as the delimiter, take care to use the
regexp \code{" +"}, not \code{" *"}.

% ----------------------------------------------------------------------------
\defineentry{pregexp-replace}
\begin{procedure}
  \code{(pregexp-replace \var{pat} \var{str} \var{ins})}
\end{procedure}
\returns{} a string

The \code{pregexp-replace} procedure replaces the matched portion of
the text string by another string.  The first argument is the pattern
\var{pat}, the second the text string \var{str}, and the third is the
string to be inserted \var{ins}, which may contain back-references
(see \S\ref{sec:pregexp-back-references}).

If the pattern doesn't occur in the text string, the returned string
is identical (\code{eq?}) to \var{str}.

% ----------------------------------------------------------------------------
\defineentry{pregexp-replace*}
\begin{procedure}
  \code{(pregexp-replace* \var{pat} \var{str} \var{ins})}
\end{procedure}
\returns{} a string

The \code{pregexp-replace*} procedure replaces \emph{all} matches of
regexp \var{pat} in the text string \var{str} by the insert string
\var{ins}, which may contain back-references (see
\S\ref{sec:pregexp-back-references}).

As with \code{pregexp-replace}, if the pattern doesn't occur in the
text string, the returned string is identical (\code{eq?}) to
\var{str}.

% ----------------------------------------------------------------------------
\defineentry{pregexp-quote}
\begin{procedure}
  \code{(pregexp-quote \var{str})}
\end{procedure}
\returns{} a U-regexp

The \code{pregexp-quote} procedure takes an arbitrary string \var{str}
and returns a U-regexp string that precisely represents it.  In
particular, characters in the input string that could serve as regexp
metacharacters are escaped with a backslash, so that they safely match
only themselves.

\code{pregexp-quote} is useful when building a composite regexp from a
mix of regexp strings and verbatim strings.

\section {The Regexp Pattern Language}

\subsection {Basic Assertions}

The \emph{assertions} \code{\textasciicircum} and \code{\$} identify
the beginning and the end of the text string respectively.  They
ensure that their adjoining regexps match at the beginning or end of
the text string.  Examples:

\code{(pregexp-match-positions "{\textasciicircum}contact" "first contact")}
$\Rightarrow$ \code{\#f}

The regexp fails to match because \code{contact} does not occur at the
beginning of the text string.

\code{(pregexp-match-positions "laugh\$" "laugh laugh laugh laugh")}
$\Rightarrow$ \code{((18 . 23))}.

The regexp matches the \emph{last} \code{laugh}.

The metasequence \code{{\textbackslash}b} asserts that a \emph{word
  boundary} exists.

\code{(pregexp-match-positions "yack{\textbackslash\textbackslash}b"
  "yackety yack")} $\Rightarrow$ \code{((8 . 12))}

The \code{yack} in \code{yackety} doesn't end at a word boundary so it
isn't matched.  The second \code{yack} does and is.

The metasequence \code{{\textbackslash}B} has the opposite effect to
\code{{\textbackslash}b}.  It asserts that a word boundary does not
exist.

\code{(pregexp-match-positions "an{\textbackslash\textbackslash}B" "an
  analysis")} $\Rightarrow$ \code{((3 . 5))}

The \code{an} that doesn't end in a word boundary is matched.

\subsection {Characters and Character Classes}

Typically a character in the regexp matches the same character in the
text string.  Sometimes it is necessary or convenient to use a regexp
metasequence to refer to a single character.  Thus, metasequences
\code{{\textbackslash}n}, \code{{\textbackslash}r},
\code{{\textbackslash}t}, and \code{{\textbackslash}.}  match the
newline, return, tab, and period characters respectively.

The \emph{metacharacter} period (\code{.}) matches \emph{any}
character other than newline.

\code{(pregexp-match "p.t" "pet")} $\Rightarrow$ \code{("pet")}

It also matches \code{pat}, \code{pit}, \code{pot}, \code{put}, and
\code{p8t} but not \code{peat} or \code{pfffft}.

A \emph{character class} matches any one character from a set of
characters. A typical format for this is the \emph{bracketed character
  class} \code{[}\etc\code{]}, which matches any one character from
the non-empty sequence of characters enclosed within the
brackets.\footnote{Requiring a bracketed character class to be
  non-empty is not a limitation, since an empty character class can be
  more easily represented by an empty string.}  Thus
\code{"p[aeiou]t"} matches \code{pat}, \code{pet}, \code{pit},
\code{pot}, \code{put} and nothing else.

Inside the brackets, a hyphen (\code{-}) between two characters
specifies the ASCII range between the characters.  For example,
\code{"ta[b-dgn-p]"} matches \code{tab}, \code{tac}, \code{tad},
\emph{and} \code{tag}, \emph{and} \code{tan}, \code{tao}, \code{tap}.

An initial caret (\code{\textasciicircum}) after the left bracket
inverts the set specified by the rest of the contents, i.e., it
specifies the set of characters \emph{other than} those identified in
the brackets.  For example, \code{"do[{\textasciicircum}g]"} matches
all three-character sequences starting with \code{do} except
\code{dog}.

Note that the metacharacter \code{\textasciicircum} inside brackets
means something quite different from what it means outside.  Most
other metacharacters (\code{.}, \code{*}, \code{+}, \code{?}, etc.)
cease to be metacharacters when inside brackets, although you may
still escape them for peace of mind.  \code{-} is a metacharacter only
when it's inside brackets, and neither the first nor the last
character.

Bracketed character classes cannot contain other bracketed character
classes (although they contain certain other types of character
classes---see below).  Thus a left bracket (\code{[}) inside a
bracketed character class doesn't have to be a metacharacter; it can
stand for itself.  For example, \code{"[a[b]"} matches \code{a},
\code{[}, and \code{b}.

Furthermore, since empty bracketed character classes are disallowed, a
right bracket (\code{]}) immediately occurring after the opening left
bracket also doesn't need to be a metacharacter.  For example,
\code{"[]ab]"} matches \code{]}, \code{a}, and \code{b}.

\subsubsection {Some Frequently Used Character Classes}

Some standard character classes can be conveniently represented as
metasequences instead of as explicit bracketed
expressions. \code{{\textbackslash}d} matches a digit using
\code{char-numeric?}; \code{{\textbackslash}s} matches a whitespace
character using \code{char-whitespace?}; and \code{{\textbackslash}w}
matches a character that could be part of a word.\footnote{Following
  regexp custom, we identify word characters as alphabetic, numeric,
  or underscore (\code{\_}).}

The upper-case versions of these metasequences stand for the
inversions of the corresponding character classes. Thus
\code{{\textbackslash}D} matches a non-digit, \code{{\textbackslash}S}
a non-whitespace character, and \code{{\textbackslash}W} a non-word
character.

Remember to include a double backslash when putting these
metasequences in a Scheme string:

\code{(pregexp-match
  "{\textbackslash\textbackslash}d{\textbackslash\textbackslash}d" "0
  dear, 1 have 2 read catch 22 before 9")} $\Rightarrow$ \code{("22")}

These character classes can be used inside a bracketed expression.
For example, \code{"[a-z{\textbackslash\textbackslash}d]"} matches a
lower-case letter or a digit.

\subsubsection {POSIX Character Classes}

A \emph{POSIX character class} is a special metasequence of the form
\code{[:}\etc\code{:]} that can be used only inside a bracketed
expression. The POSIX classes supported are:

\begin{center}\begin{tabular}{ll}
\code{[:alnum:]}  & letters and digits \\
\code{[:alpha:]}  & letters \\
\code{[:algor:]}  & the letters \code{c}, \code{h}, \code{a} and \code{d} \\
\code{[:ascii:]}  & 7-bit ASCII characters \\
\code{[:blank:]}  & widthful whitespace, i.e., space and tab \\
\code{[:cntrl:]}  & control characters, viz, those with code $< 32$ \\
\code{[:digit:]}  & digits, same as \code{{\textbackslash}d} \\
\code{[:graph:]}  & characters that use ink \\
\code{[:lower:]}  & lower-case letters \\
\code{[:print:]}  & ink-users plus widthful whitespace \\
\code{[:space:]}  & whitespace, same as \code{{\textbackslash}s} \\
\code{[:upper:]}  & upper-case letters \\
\code{[:word:]}   & letters, digits, and underscore, same as \code{{\textbackslash}w} \\
\code{[:xdigit:]} & hex digits \\
\end{tabular}\end{center}

For example, the regexp \code{"[[:alpha:]\_]"} matches a letter or
underscore.

\code{(pregexp-match "[[:alpha:]\_]" "--x--")} $\Rightarrow$ \code{("x")}

\code{(pregexp-match "[[:alpha:]\_]" "--\_--")} $\Rightarrow$
\code{("\_")}

\code{(pregexp-match "[[:alpha:]\_]" "--:--")} $\Rightarrow$ \code{\#f}

The POSIX class notation is valid \emph{only} inside a bracketed
expression.  For instance, \code{[:alpha:]}, when not inside a
bracketed expression, will \emph{not} be read as the letter class.
Rather it is (from previous principles) the character class containing
the characters \code{:}, \code{a}, \code{l}, \code{p}, and \code{h}.

\code{(pregexp-match "[:alpha:]" "--a--")} $\Rightarrow$ \code{("a")}

\code{(pregexp-match "[:alpha:]" "--\_--")} $\Rightarrow$ \code{\#f}

By placing a caret (\code{\textasciicircum}) immediately after
\code{[:}, you get the inversion of that POSIX character class.  Thus,
  \code{[:{\textasciicircum}alpha:]} is the class containing all
  characters except the letters.

\subsection {Quantifiers}

The \emph{quantifiers} \code{*}, \code{+}, and
\code{?} match respectively: zero or more, one or more,
and zero or one instances of the preceding subpattern.

\code{(pregexp-match-positions "c[ad]*r" "cadaddadddr")} $\Rightarrow$
\code{((0 . 11))}

\code{(pregexp-match-positions "c[ad]*r" "cr")} $\Rightarrow$
\code{((0 . 2))}

\code{(pregexp-match-positions "c[ad]+r" "cadaddadddr")} $\Rightarrow$
\code{((0 . 11))}

\code{(pregexp-match-positions "c[ad]+r" "cr")} $\Rightarrow$
\code{\#f}

\code{(pregexp-match-positions "c[ad]?r" "cadaddadddr")} $\Rightarrow$
\code{\#f}

\code{(pregexp-match-positions "c[ad]?r" "cr")} $\Rightarrow$
\code{((0 . 2))}

\code{(pregexp-match-positions "c[ad]?r" "car")} $\Rightarrow$
\code{((0 . 3))}

\subsubsection {Numeric Quantifiers}

You can use braces to specify much finer-tuned quantification than is
possible with \code{*}, \code{+}, and \code{?}.

The quantifier \code{\{\var{m}\}} matches exactly \var{m} instances of
the preceding subpattern.  \var{m} must be a nonnegative integer.

The quantifier \code{\{\var{m},\var{n}\}} matches at least \var{m} and
at most \var{n} instances.  \var{m} and \var{n} are nonnegative
integers with $\var{m} \le \var{n}$.  You may omit either or both
numbers, in which case \var{m} defaults to 0 and \var{n} to infinity.

It is evident that \code{+} and \code{?} are abbreviations for
\code{\{1,\}} and \code{\{0,1\}} respectively.  \code{*} abbreviates
\code{\{,\}}, which is the same as \code{\{0,\}}.

\code{(pregexp-match "[aeiou]\{3\}" "vacuous")} $\Rightarrow$
\code{("uou")}

\code{(pregexp-match "[aeiou]\{3\}" "evolve")} $\Rightarrow$ \code{\#f}

\code{(pregexp-match "[aeiou]\{2,3\}" "evolve")} $\Rightarrow$
\code{\#f}

\code{(pregexp-match "[aeiou]\{2,3\}" "zeugma")} $\Rightarrow$
\code{("eu")}

\subsubsection {Non-greedy Quantifiers}

The quantifiers described above are \emph{greedy}, i.e., they match
the maximal number of instances that would still lead to an overall
match for the full pattern.

\code{(pregexp-match "<.*>" "<tag1> <tag2> <tag3>")} $\Rightarrow$
\code{("<tag1> <tag2> <tag3>")}

To make these quantifiers \emph{non-greedy}, append a \code{?} to
them.  Non-greedy quantifiers match the minimal number of instances
needed to ensure an overall match.

\code{(pregexp-match "<.*?>" "<tag1> <tag2> <tag3>")} $\Rightarrow$
\code{("<tag1>")}

The non-greedy quantifiers are respectively: \code{*?}, \code{+?},
\code{??}, \code{\{\var{m}\}?}, and \code{\{\var{m},\var{n}\}?}.  Note
the two uses of the metacharacter \code{?}.

\subsection {Clusters}

\emph{Clustering}, i.e., enclosure within parentheses
\code{(}\etc\code{)}, identifies the enclosed subpattern as a single
entity.  It causes the matcher to \emph{capture} the \emph{submatch},
or the portion of the string matching the subpattern, in addition to
the overall match.

\code{(pregexp-match "([a-z]+) ([0-9]+), ([0-9]+)" "jan 1, 1970")} \\
$\Rightarrow$ \code{("jan 1, 1970" "jan" "1" "1970")}

Clustering also causes a following quantifier to treat the entire
enclosed subpattern as an entity.

\code{(pregexp-match "(poo )*" "poo poo platter")} $\Rightarrow$
\code{("poo poo " "poo ")}

The number of submatches returned is always equal to the number of
subpatterns specified in the regexp, even if a particular subpattern
happens to match more than one substring or no substring at all.

\code{(pregexp-match "([a-z ]+;)*" "lather; rinse; repeat;")} \\
$\Rightarrow$ \code{("lather; rinse; repeat;" " repeat;")}

Here the \code{*}-quantified subpattern matches three times, but it is
the last submatch that is returned.

It is also possible for a quantified subpattern to fail to match, even
if the overall pattern matches.  In such cases, the failing submatch
is represented by \code{\#f}.

\begin{alltt}
(define date-re
  ;; match 'month year' or 'month day, year'.
  ;; subpattern matches day, if present
  (pregexp "([a-z]+) +([0-9]+,)? *([0-9]+)"))
\end{alltt}

\code{(pregexp-match date-re "jan 1, 1970")} $\Rightarrow$
\code{("jan 1, 1970" "jan" "1," "1970")}

\code{(pregexp-match date-re "jan 1970")} $\Rightarrow$
\code{("jan 1970" "jan" \#f "1970")}

\subsubsection {Back-references}\label{sec:pregexp-back-references}

Submatches can be used in the insert string argument of the procedures
\code{pregexp-replace} and \code{pregexp-replace*}.  The insert string
can use \code{\textbackslash}$n$ as a \emph{back-reference} to refer
back to the $n^\textrm{th}$ submatch, i.e., the substring that matched
the $n^\textrm{th}$ subpattern.  \code{{\textbackslash}0} refers to
the entire match, and it can also be specified as
\code{\textbackslash\&}.

\code{(pregexp-replace "\_(.+?)\_" "the \_nina\_, the \_pinta\_, and
  the \_santa maria\_" "*{\textbackslash\textbackslash}1*")}
$\Rightarrow$ \code{"the *nina*, the \_pinta\_, and the \_santa
  maria\_"}

\code{(pregexp-replace* "\_(.+?)\_" "the \_nina\_, the \_pinta\_, and
  the \_santa maria\_" "*{\textbackslash\textbackslash}1*")}
$\Rightarrow$ \code{"the *nina*, the *pinta*, and the *santa maria*"}

\code{(pregexp-replace "({\textbackslash\textbackslash}S+)
  ({\textbackslash\textbackslash}S+)
  ({\textbackslash\textbackslash}S+)" "eat to live"
  "{\textbackslash\textbackslash}3 {\textbackslash\textbackslash}2
  {\textbackslash\textbackslash}1")} \\
$\Rightarrow$ \code{"live to eat"}

Use \code{\textbackslash\textbackslash} in the insert string to
specify a literal backslash.  Also, \code{\textbackslash\$} stands for
an empty string, and is useful for separating a back-reference
\code{\textbackslash\var{n}} from an immediately following number.

Back-references can also be used within the regexp pattern to refer
back to an already matched subpattern in the pattern.
\code{\textbackslash}$n$ stands for an exact repeat of the
$n^\textrm{th}$ submatch.\footnote{\code{{\textbackslash}0}, which is
  useful in an insert string, makes no sense within the regexp
  pattern, because the entire regexp has not matched yet that you
  could refer back to it.}

\code{(pregexp-match "([a-z]+) and {\textbackslash\textbackslash}1"
  "billions and billions")} \\
$\Rightarrow$ \code{("billions and billions" "billions")}

Note that the back-reference is not simply a repeat of the previous
subpattern.  Rather it is a repeat of \emph{the particular substring
  already matched by the subpattern}.

In the above example, the back-reference can only match
\code{billions}.  It will not match \code{millions}, even though the
subpattern it harks back to---\code{([a-z]+)}---would have had no
problem doing so:

\code{(pregexp-match "([a-z]+) and {\textbackslash\textbackslash}1"
  "billions and millions")} $\Rightarrow$ \code{\#f}

The following corrects doubled words:

\code{(pregexp-replace* "({\textbackslash\textbackslash}S+)
  {\textbackslash\textbackslash}1" "now is the the time for all good
  men to to come to the aid of of the party"
  "{\textbackslash\textbackslash}1")} \\
$\Rightarrow$ \code{"now is the time for all good men to come to the
  aid of the party"}

The following marks all immediately repeating patterns in a number
string:

\code{(pregexp-replace*
  "({\textbackslash\textbackslash}d+){\textbackslash\textbackslash}1"
  "123340983242432420980980234"
  "{{\textbackslash\textbackslash}1,{\textbackslash\textbackslash}1}")}
\\ $\Rightarrow$ \code{"12{3,3}40983{24,24}3242{098,098}0234"}

\subsubsection {Non-capturing Clusters}

It is often required to specify a cluster (typically for
quantification) but without triggering the capture of submatch
information.  Such clusters are called \emph{non-capturing}.  In such
cases, use \code{(?:} instead of \code{(} as the cluster opener.  In
the following example, the non-capturing cluster eliminates the
directory portion of a given pathname, and the capturing cluster
identifies the basename.

\code{(pregexp-match "\textasciicircum(?:[a-z]*/)*([a-z]+)\$"
  "/usr/local/bin/scheme")} \\
$\Rightarrow$ \code{("/usr/local/bin/scheme" "scheme")}

\subsubsection {Cloisters}

The location between the \code{?} and the \code{:} of a non-capturing
cluster is called a \emph{cloister}.\footnote{A useful, if terminally
  cute, coinage from the abbots of Perl~\cite{pperl}.}  You can put
\emph{modifiers} there that will cause the enclustered subpattern to
be treated specially.  The modifier \code{i} causes the subpattern to
match \emph{case-insensitively}:

\code{(pregexp-match "(?i:hearth)" "HeartH")} $\Rightarrow$
\code{("HeartH")}

The modifier \code{x} causes the subpattern to match
\emph{space-insensitively}, i.e., spaces and comments within the
subpattern are ignored.  Comments are introduced as usual with a
semicolon (\code{;}) and extend till the end of the line.  If you need
to include a literal space or semicolon in a space-insensitized
subpattern, escape it with a backslash.

\code{(pregexp-match "(?x: a lot)" "alot")} $\Rightarrow$
\code{("alot")}

\code{(pregexp-match "(?x: a  {\textbackslash\textbackslash}  lot)" "a
  lot")} \\
$\Rightarrow$ \code{("a lot")}

\begin{alltt}
(pregexp-match "(?x:
   a {\textbackslash\textbackslash} man  {\textbackslash\textbackslash}; {\textbackslash\textbackslash}   ; ignore
   a {\textbackslash\textbackslash} plan {\textbackslash\textbackslash}; {\textbackslash\textbackslash}   ; me
   a {\textbackslash\textbackslash} canal         ; completely
   )"
 "a man; a plan; a canal")
\end{alltt}\antipar
$\Rightarrow$ \code{("a man; a plan; a canal")}

You can put more than one modifier in the cloister.

\begin{alltt}
(pregexp-match "(?ix:
   a {\textbackslash\textbackslash} man  {\textbackslash\textbackslash}; {\textbackslash\textbackslash}   ; ignore
   a {\textbackslash\textbackslash} plan {\textbackslash\textbackslash}; {\textbackslash\textbackslash}   ; me
   a {\textbackslash\textbackslash} canal         ; completely
   )"
 "A Man; a Plan; a Canal")
\end{alltt}\antipar
$\Rightarrow$ \code{("A Man; a Plan; a Canal")}

A minus sign before a modifier inverts its meaning.  Thus, you can use
\code{-i} and \code{-x} in a {\em subcluster} to overturn the
insensitivities caused by an enclosing cluster.

\code{(pregexp-match "(?i:the (?-i:TeX)book)" "The TeXbook")}
$\Rightarrow$ \code{("The TeXbook")}

This regexp will allow any casing for \code{the} and \code{book} but
insists that \code{TeX} not be differently cased.

\subsection {Alternation}\label{sec:pregexp-alternation}

You can specify a list of \emph{alternate} subpatterns by separating
them by \code{|}.  The \code{|} separates subpatterns in the nearest
enclosing cluster (or in the entire pattern string if there are no
enclosing parentheses).

\code{(pregexp-match "f(ee|i|o|um)" "a small, final fee")}
$\Rightarrow$ \code{("fi" "i")}

\begin{alltt}
(pregexp-replace* "([yi])s(e[sdr]?|ing|ation)"
 "it is energising to analyse an organisation pulsing with noisy organisms"
 "{\textbackslash\textbackslash}1z{\textbackslash\textbackslash}2")
\end{alltt}\antipar
$\Rightarrow$ \code{"it is energizing to analyze an organization
  pulsing with noisy organisms"}

Note again that if you wish to use clustering merely to specify a list
of alternate subpatterns but do not want the submatch, use \code{(?:}
instead of \code{(}.

\code{(pregexp-match "f(?:ee|i|o|um)" "fun for all")} $\Rightarrow$
\code{("fo")}

An important thing to note about alternation is that the leftmost
matching alternate is picked regardless of its length.  Thus, if one
of the alternates is a prefix of a later alternate, the latter may not
have a chance to match.

\code{(pregexp-match "call|call/cc" "call/cc")} $\Rightarrow$
\code{("call")}

To allow the longer alternate to have a shot at matching, place it
before the shorter one:

\code{(pregexp-match "call/cc|call" "call/cc")} $\Rightarrow$
\code{("call/cc")}

In any case, an overall match for the entire regexp is always
preferred to an overall non-match.  In the following, the longer
alternate still wins, because its preferred shorter prefix fails to
yield an overall match.

\code{(pregexp-match "(?:call|call/cc) constrained" "call/cc
  constrained")} \\
$\Rightarrow$ \code{("call/cc constrained")}

\subsection {Backtracking}

We've already seen that greedy quantifiers match the maximal number of
times, but the overriding priority is that the overall match succeed.
Consider

\code{(pregexp-match "a*a" "aaaa")}

The regexp consists of two subregexps, \code{a*} followed by \code{a}.
The subregexp \code{a*} cannot be allowed to match all four \code{a}'s
in the text string \code{"aaaa"}, even though \code{*} is a greedy
quantifier.  It may match only the first three, leaving the last one
for the second subregexp.  This ensures that the full regexp matches
successfully.

The regexp matcher accomplishes this via a process called
\emph{backtracking}.  The matcher tentatively allows the greedy
quantifier to match all four \code{a}'s, but then when it becomes
clear that the overall match is in jeopardy, it \emph{backtracks} to a
less greedy match of {\em three} \code{a}'s.  If even this fails, as
in the call

\code{(pregexp-match "a*aa" "aaaa")}

the matcher backtracks even further.  Overall failure is conceded only
when all possible backtracking has been tried with no success.

Backtracking is not restricted to greedy quantifiers.  Nongreedy
quantifiers match as few instances as possible, and progressively
backtrack to more and more instances in order to attain an overall
match.  There is backtracking in alternation, too, as the more
rightward alternates are tried when locally successful leftward ones
fail to yield an overall match.

\subsubsection {Disabling Backtracking}

Sometimes it is efficient to disable backtracking.  For example, we
may wish to \emph{commit} to a choice, or we know that trying
alternatives is fruitless.  A non-backtracking regexp is enclosed in
\code{(?>}\etc\code{)}.

\code{(pregexp-match "(?>a+)." "aaaa")} $\Rightarrow$ \code{\#f}

In this call, the subregexp \code{?>a+} greedily matches all four
\code{a}'s, and is denied the opportunity to backtrack.  So the
overall match is denied.  The effect of the regexp is therefore to
match one or more \code{a}'s followed by something that is definitely
non-\code{a}.

\subsection {Looking Ahead and Behind}

You can have assertions in your pattern that look {\em ahead} or
\emph{behind} to ensure that a subpattern does or does not occur.
These look-around assertions are specified by putting the subpattern
checked for in a cluster whose leading characters are \code{?=} for
positive look-ahead, \code{?!} for negative look-ahead, \code{?<=} for
positive look-behind, and \code{?<!} for negative look-behind.  Note
that the subpattern in the assertion does not generate a match in the
final result.  It merely allows or disallows the rest of the match.

\subsubsection {Look-ahead}

Positive look-ahead (\code{?=}) peeks ahead to ensure that its
subpattern \emph{could} match.

\code{(pregexp-match-positions "grey(?=hound)"
  "i left my grey socks at the greyhound")} \\
$\Rightarrow$ \code{((28 . 32))}

The regexp \code{"grey(?=hound)"} matches \code{grey}, but \emph{only}
if it is followed by \code{hound}.  Thus, the first \code{grey} in the
text string is not matched.

Negative look-ahead (\code{?!}) peeks ahead to ensure that its
subpattern could not possibly match.

\code{(pregexp-match-positions "grey(?!hound)"
  "the gray greyhound ate the grey socks")}\\
$\Rightarrow$ \code{((27 . 31))}

The regexp \code{"grey(?!hound)"} matches \code{grey}, but only if it
is \emph{not} followed by \code{hound}.  Thus the \code{grey} just
before \code{socks} is matched.

\subsubsection {Look-behind}

Positive look-behind (\code{?<=}) checks that its subpattern
\emph{could} match immediately to the left of the current position in
the text string.

\code{(pregexp-match-positions "(?<=grey)hound"
  "the hound is not a greyhound")} \\
$\Rightarrow$ \code{((23 . 28))}

The regexp \code{(?<=grey)hound} matches \code{hound}, but only if it
is preceded by \code{grey}.

Negative look-behind (\code{?<!}) checks that its subpattern could not
possibly match immediately to the left.

\code{(pregexp-match-positions "(?<!grey)hound"
  "the greyhound is not a hound")} \\
$\Rightarrow$ \code{((23 . 28))}

The regexp \code{(?<!grey)hound} matches \code{hound}, but only if it
is \emph{not} preceded by \code{grey}.

Look-aheads and look-behinds can be convenient when they are not
confusing.
