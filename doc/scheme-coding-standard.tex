% Copyright 2018 Beckman Coulter, Inc.
%
% Permission is hereby granted, free of charge, to any person
% obtaining a copy of this software and associated documentation files
% (the "Software"), to deal in the Software without restriction,
% including without limitation the rights to use, copy, modify, merge,
% publish, distribute, sublicense, and/or sell copies of the Software,
% and to permit persons to whom the Software is furnished to do so,
% subject to the following conditions:
%
% The above copyright notice and this permission notice shall be
% included in all copies or substantial portions of the Software.
%
% THE SOFTWARE IS PROVIDED "AS IS", WITHOUT WARRANTY OF ANY KIND,
% EXPRESS OR IMPLIED, INCLUDING BUT NOT LIMITED TO THE WARRANTIES OF
% MERCHANTABILITY, FITNESS FOR A PARTICULAR PURPOSE AND
% NONINFRINGEMENT. IN NO EVENT SHALL THE AUTHORS OR COPYRIGHT HOLDERS
% BE LIABLE FOR ANY CLAIM, DAMAGES OR OTHER LIABILITY, WHETHER IN AN
% ACTION OF CONTRACT, TORT OR OTHERWISE, ARISING FROM, OUT OF OR IN
% CONNECTION WITH THE SOFTWARE OR THE USE OR OTHER DEALINGS IN THE
% SOFTWARE.

\documentclass[letterpaper,11pt,twoside,final]{article}
\usepackage{sagian}

\begin{document}

\title {Scheme Coding Standard}
\author {Chris Payne \and Bob Burger}
\date {\copyright\ 2018 Beckman Coulter, Inc.
  Licensed under the \href{https://opensource.org/licenses/MIT}{MIT License}.}
\coverpage

\section* {Purpose}

This document describes the coding standard for Scheme code.  It is
heavily inspired by the Erlang Programming Rules and
Conventions~\cite{erlang-programming-rules}, as well as \emph{The
  Scheme Programming Language}~\cite{the-scheme-programming-language}.

\section* {General}

\subsection* {Avoid set!}

The primitive procedures \code{set!} and \code{vector-set!} should
be avoided. If mutation is necessary, consider using processes,
hashtables, or dictionaries to help abstract the details into an easy
to reason about form.

Don't do this:
\antipar
\begin{alltt}
(define (sum ls)
  (let ([acc 0])
    (for-each
     (lambda (x)
       (set! acc (+ acc x)))
     ls)
    acc))
\end{alltt}

A better form without \code{set!} is:
\antipar
\begin{alltt}
(define (sum ls)
  (fold-left
   (lambda (acc x)
     (+ acc x))
   0
   ls))
\end{alltt}

\subsection* {Use structured data instead of strings}

Use structured data to communicate to other parts of the system. Using
strings directly makes localization much harder.

Don't do this:
\antipar
\begin{alltt}
(format "Homing stalled for motor ~a at step ~d" name step)
\end{alltt}

Do this:
\antipar
\begin{alltt}
`\#(home-stalled ,name ,step)
\end{alltt}

Symbols can be used in structured data. Structured data can be pattern
matched and converted to strings in an appropriate language.

\subsection* {Do not program defensively}

According to the Erlang Programming Rules and Conventions~\cite{erlang-programming-rules}:

\begin{quotation}
\noindent
A defensive program is one where the programmer does not ``trust'' the input
data to the part of the system they are programming. In general one should not
test input data to functions for correctness. Most of the code in the system
should be written with the assumption that the input data to the function in
question is correct. Only a small part of the code should actually perform any
checking of the data. This is usually done when data ``enters'' the system for
the first time, once data has been checked as it enters the system it should
thereafter be assumed correct.
\end{quotation}

\subsection* {Think about append}

\code{append} will copy all of its arguments except the last. Its
time and space complexity are O($n$) for a list of $n$ elements.

O($n$) is likely fine in some cases, but repeated calls to
\code{append} will result in O($n^2$) or worse. Here is an example
that could be written better without \code{append}.
\antipar
\begin{alltt}
(let lp ([ls (list 1 2 3)] [seen '()])
  (match ls
    [() seen]
    [(,first . ,rest)
     (lp rest (append seen (list first)))]))
\end{alltt}

This version of the code is O($n$) to traverse the list, uses
\code{cons} to build up a result, then uses \code{reverse} which
is also O($n$) to return the result in the proper order.
\antipar
\begin{alltt}
(let lp ([ls (list 1 2 3)] [seen '()])
  (match ls
    [() (reverse seen)]
    [(,first . ,rest)
     (lp rest (cons first seen))]))
\end{alltt}

This is really a more general rule about knowing what the functions
you are calling do. Understand their performance characteristics and
know that they are the right functions to use.

\subsection* {Use match instead of car and cdr}

Over time, we have found \code{match} to be simple to read. It is
so nice that we began trying to minimize our direct usage of
\code{car} and \code{cdr}.

Without \code{match}:
\antipar
\begin{alltt}
(define (fold f acc ls)
  (if (null? ls)
      acc
      (fold f (f (car ls) acc) (cdr ls))))
\end{alltt}

With \code{match}:
\antipar
\begin{alltt}
(define (fold f acc ls)
  (match ls
    [() acc]
    [(,first . ,rest) (fold f (f first acc) rest)]))
\end{alltt}

\subsection* {Use match or match-let* with tuples}

When binding multiple fields from a tuple, use \code{match},
\code{match-define}, or
\code{match-let*} to minimize type checking.

Given:
\antipar
\begin{alltt}
(define-tuple <point> x y z)
\end{alltt}

Don't extract each field:
\antipar
\begin{alltt}
(lambda (p)
  (let ([x (<point> x p)]
        [y (<point> y p)]
        [z (<point> z p)])
    (list x y z)))
\end{alltt}

Extract all fields at once:
\antipar
\begin{alltt}
(lambda (p)
  (match-let* ([`(<point> [x ,x] [y ,y] [z ,z]) p])
    (list x y z)))
\end{alltt}

\subsection* {Fail fast}

When dealing with code that uses exceptions, use a fail-fast approach
for the code. This usually produces code that is easier to read and
reason about.
It may also provide more informative debugging context by putting
the call to \code{throw} in non-tail position, with variables of
interest live after the call.

Don't do this:
\antipar
\begin{alltt}
(if \var{condition}
    (success)
    (throw 'boom!))
\end{alltt}

Do this:
\antipar
\begin{alltt}
(unless \var{condition}
  (throw 'boom!))
(success)
\end{alltt}

If you must raise an exception, prefer \code{throw} to \code{raise} in
situations where the current continuation may provide useful debugging
context.

If you must trap exceptions, use the \code{try} form since it preserves
debugging context that is discarded by the older \code{catch} form.

To examine trapped exceptions, use the \code{catch} match extension since
it is compatible with both \code{try} and \code{catch}.
For example, the definition of \code{g} below works with either
path through \code{f}.
\codebegin
(define (f x)
  (if (eq? x 'old-school)
      (catch \(e\sb{0}\) \etc{} \(e\sb{n}\))
      (try \(e\sb{0}\) \etc{} \(e\sb{n}\))))
(define (g x)
  (match (f x)
    [`(catch fire ,e) (throw 'water e)]
    [`(catch ,_ ,e) (throw e)]
    \etc))
\codeend

\subsection* {Know when to use \texttt{eq?}, \texttt{eqv?},
  \texttt{equal?}, and \texttt{=}}

Chapter 6 of \emph{The Scheme Programming
  Language}~\cite{the-scheme-programming-language} discusses
equivalence.

Use \code{eq?} to check if two objects are represented by the same
pointer value.

Use \code{eqv?} to check if two objects are equivalent. This works
with simple data like numbers, booleans, and characters.

Use \code{equal?} to check if two objects have the same
structure. This works with compound data like strings, lists, and
vectors.

Use \code{=} to check if two numbers are equal.

\subsection* {Tag messages}

Messages should be tagged. This makes the order of \code{receive}
and \code{match} statements less important and the implementation of
new messages easier.

Don't do this:
\antipar
\begin{alltt}
(define (loop)
  (receive
   [(,msg ,args)
    (apply printf msg args)
    (loop)]))
\end{alltt}

Do this:
\antipar
\begin{alltt}
(define (loop)
  (receive
   [\#(print ,msg ,args)
    (apply printf msg args)
    (loop)]))
\end{alltt}

\subsection* {Libraries}

The filename and library name should be consistent. The export and
import lists should be sorted. The \code{except} clauses should be
at the end of the \code{import} list.

In Emacs, select multiple lines and use Meta-X \code{sort-lines}.

For example in \texttt{swish/gen-server.ss} we find the
\code{(swish gen-server)} library: \antipar
\begin{alltt}
(library (swish gen-server)
  (export
   define-state-tuple
   gen-server:call
   gen-server:cast
   gen-server:debug
   gen-server:reply
   gen-server:start
   gen-server:start&link
   )
  (import
   (chezscheme)
   (swish erlang)
   (swish event-mgr-notify)
   (swish events)
   )
\end{alltt}

\section* {Concurrency}

\subsection* {Input and Output}

When choosing which input/output functions to use, prefer the ones
provided by the concurrency subsystem. Chez Scheme's builtin
operations may prevent the entire system from functioning for a period
of time. Our versions of the functions will initiate asynchronous read
and write operations, then continue processing other activities.

\subsection* {Event-loop Callbacks}

Inspect event-loop callbacks with great scrutiny. Well-designed
callbacks typically register objects that wrap operating system
interface handles with a guardian and send messages to a
process. These types of operations neither block nor fail.

\subsection* {Finalizers}

Ill-behaved finalizers may cause memory and handle leaks. Inspect
finalizers with great scrutiny. Well-designed finalizers guard against
errors when closing handles and include time-outs so that they don't
wait indefinitely.

\subsection* {\texttt{gen-server}}

In \code{handle-call}, \code{handle-cast}, and
\code{handle-info}, do not use specific match clauses to guard
against unexpected messages. An unexpected message will cause the
server to crash, and the framework will report the failure.

\section* {Style}

Chapter~1 of \emph{The Scheme Programming Language} provides a good
starting point for writing Scheme code.

\subsection* {Only bind variables you use}

\subsection* {Identifiers}

Use meaningful names---this can be very difficult. Identifiers should
correspond directly to the names used in the design
documents. Identifiers should use lower-case letters with a hyphen
separating compound words. Mixed-case identifiers may be used when
they directly correspond to an external definition.

\subsection* {Comments}

Comments should be clear, concise, and written in U.S. English.

Avoid using comments that say what the code says.

\subsection* {Formatting}

Scheme code should be indented according to the Emacs Scheme mode
(\texttt{scheme.el}). Ctrl-Meta-Q in Emacs will reindent an
expression.

Line length should be limited to 79 characters. Lines longer than 79
characters may be tolerated.

Lines should end in LF, not CRLF.

Tab characters should not appear in Scheme code. Their presence in a
file makes the indentation depend on the software used to view the
file.

\bibliography{reference}

\end{document}
